\mainsection{SCALE Dummies Guide}
There are a lot of types defined here, and to exploit the most
from the system you need to (kind of) understand how the MPC
engine works under-the-hood. As most people will not be able
to do that we here provide a quick dummy guide to get the most
performance out of the system.
In this section we are mainly talking about your programs, and
not system tuning for an application. The latter topic is far
more of a black-art alas.

\subsection{Integer Types}
There two {\em basic} clear integer types $\ints$ and \verb|ClearModp|.
The former represent $64$-bit integers and the latter integers
modulo $p$, for the prime you have chosen. These work at
roughly machine speed, bar any overhead due to the VM.
There is an {\em advanced} clear integer types \verb|Integer<K>|
which holds an integer in $\Z$ as a \verb|ClearModp| but with the 
`soft' gaurantee that the integer lies in $\Zk$.
On the secret side these are duplicated by the \verb|SecretI64|,
\verb|SecretModp| and \verb|SecretInteger<K>| types.

Note the secret operations are {\em always} more expensive than
the clear operations, and that \verb|SecretI64| operations are
generally more expensive than \verb|SecretModp| operations.
The \verb|SecretInteger<K>| type is more costly than a
\verb|SecretModp| type, as it has to worry about size issues.
Conversion between \verb|SecretModp| and  \verb|SecretI64| types
are expensive, thus best avoided if possible.
Conversion betwen \verb|SecretModp| and  \verb|SecretInteger<K>|
types if for free, as it is assumed the programmer knows
the conversion will be `valid'.

But there seems to be more arithmetic operations for \verb|SecretI64|
than for \verb|SecretModp|. For example, you cannot compare
two \verb|SecretModp| values (after all integers modulo $p$ have
no notion of `size', so mathematically this makes no sense).

However, in MPC algorithms we often use \verb|SecretModp| values
to hold integers (i.e. elements of $\Z$) and compute on them
whilst keeping a mental note of how big they are. This means
we can perform operations by keeping in the \verb|SecretModp|
domain, without needing to convert to \verb|SecretI64| values.

\begin{lstlisting}
   let a = SecretModp::from(10);
   let b = SecretModp::from(20);

   let c = a*b;
   let d = a+b;

   // Suppose we now want to "compare" c and d.
   // We know the max size is 9 bits (c is 200,
   // which is at least 9 bits in signed representation)
   let ci : SecretInteger<9> = SecretInteger::from(c);
   let di : SecretInteger<9> = SecretInteger::from(d);
   let compare = ci.lt(di);
\end{lstlisting}
The above is {\em much} faster than mapping \verb|c|
and \verb|d| over to the \verb|SecretI64| domain and doing
the comparison there.
Obviously the above code is a bit fake, as you know what the
values are, but you can see the idea.

\subsection{Floating Point Types}
In general \verb|ClearIEEE| should be your preferred clear floating
point type, as it provides almost machine level performance.
The preferred secret floating point type is \verb|SecretFixed<K,F>|
as it is much faster. However, it only provides fixed point arithmetic.
If true floating point is required then  \verb|SecretFloat<V,P>|
is better, but it can result in programs which take ages to compile.
For faster compilation, but slower programs, use  \verb|SecretIEEE|

\subsection{Bits}
There is a \verb|SecretBit| type. Due to some current limitations
of the intermediate \verb|wasm| representation there is a lot of
conversions between \verb|SecretBit| types and \verb|SecretModp|
types under the hood. This results in slower programs than are really
necessary. Once \verb|wasm| has been extended to cope with more than
four basic register types this restriction will be removed.

