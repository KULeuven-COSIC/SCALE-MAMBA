\mainsection{Dummies Guide}
Using Rust has some advantages in that we have strong typing and the
programming language has been designed to be safe in terms of stopping
programmers making errors.
However, these advantages come at a disadvantage (especially if you 
are used to Python or C++). 
So in this section we give an informal overview of how these differences
affect programming in our Rust system, and in particular with relevance
to SCALE.

\subsection{Strong Typing}
We take advantage of strong typing to ensure that the types correspond
to mathematically what they represent.
For efficiency we give access to the basic SCALE internal types
$\ints$, $\ClearModp$, $\SecretModp$, $\SecretI$ and $\SecretBit$.
But in MAMBA you could do something like
\begin{lstlisting}
    program.bit_length = 16
    program.security = 40

    a = sint(4)
    b = sint(5)
    c = a<b
\end{lstlisting}
But this is mathematically meaningless, what is meant here is that
you are going to `think' of \verb|sint| values as integers of bit
size sixteen and then do the comparison assuming that the numbers
are indeed of size bounded by $2^{16}$. After all a comparison
of values in the finite field $\F_p$ really makes no mathematical
sense.

In our Rust system we want to avoid such implicit assumptions
that a reader of your code needs to make. Thus we provide types
which help capture what you really want to do.
So the above MAMBA code would become
\begin{lstlisting}
   let a: SecretInteger<16> = SecretInteger::from(4);
   let b: SecretInteger<16> = SecretInteger::from(5);
   let c = a.lt(b);
\end{lstlisting}
The type of the value \verb|c| is a $\SecretModp$ value.
We use the member function notation \verb|a.lt(b)| instead
of \verb|a<b| to force the programmer to realise that you cannot
use the output of the comparison in an \verb|if|-statement.

Another aspect of this strong typing is the above use of the
\verb|from| command. This is used to convert one type to another,
which needs to be done explicitly in almost all case.

\subsection{Mutable vs Non-Mutable}
Common problems in programs are that people accidentally re-assign
a variable and then want the old value again. This is because in
languages like C++ or python all variables are mutable by default.
In C++ it is usually considered good practice to define all inputs
to a function to be \verb|const| if they are not going to be returned
as changed for this reason.
Rust goes one step further and assumes all variables are non-mutable 
by default. Thus you cannot do
\begin{lstlisting}
    let a = 3;
    if some_condition {
       a = a + 1;
    }
\end{lstlisting}
To enable this you explicitly have to signal that the variable
is going to be changed by writing
\begin{lstlisting}
    let mut a = 3;
    if some_condition {
       a = a + 1;
    }
\end{lstlisting}

\subsection{Container Types}
Container types are really important in programming, yet in Rust
they can be a bit confusing mainly due to the mutability issue
above, and the need to maintain safety of the language.
The \verb|Array| and \verb|Slice| types we define are very similar
to the standard \verb|Vec| type in Rust, and hopefully eventually
they will be the same.
The difference between an  \verb|Array| and a \verb|Slice|  is that
the size of an  \verb|Array| is known at compile time, whereas
a \verb|Slice| size may not.
This distinction enables us to do some optimizations.

Suppose you have an array of ten $\ClearModp$ values
\begin{lstlisting}
    let mut a: Array<ClearModp, 10> = Array::uninitialized();
\end{lstlisting}
You may want to assign values to the array elements, or use them
later. There are {\em four} ways of getting an array element:
\begin{enumerate}
  \item \verb|A.get(i)| returns a reference to the array and performs
        an out of bounds check.
  \item \verb|A.get_unchecked(i)| returns a reference to the array and 
        does not perform an out of bounds check.
  \item \verb|A.get_mut(i)| returns a mutable reference to the array and performs
        an out of bounds check.
  \item \verb|A.get_mut_unchecked(i)| returns a mutable reference to the array and 
        does not perform an out of bounds check.
\end{enumerate}
When accessing the reference it comes wrapped in an \verb|Option| when
using the checked values, thus you need to \verb|unwrap| this option before
using the item. 
The unwrapping itself produces a guarded object, to remove the \verb|Guard| you 
need to de-reference it.

Thus when you use a \verb|get|, but not a \verb|get_unchecked|, 
you need to unwrap the result, then in both cases you should de-reference,
i.e. you do
\begin{lstlisting}
   println!(" a[2] = ",*a.get(2).unwrap());
   println!(" a[2] = ",*a.get_unchecked(2));
\end{lstlisting}
However, for printing we have added some code to make the following more
syntactically nicer code work,
\begin{lstlisting}
   println!(" a[2] = ",a.get(2).unwrap());
   println!(" a[2] = ",a.get_unchecked(2));
\end{lstlisting}

If you want to access an element {\em and not change it} you should use
one of the non-mutable \verb|get| operations, if you want to change an
element you should access one of the mutable, i.e. \verb|get_mut|, operations.
This is particularly relevant when the internal object is another \verb|Array| or
\verb|Slice|.
\begin{lstlisting}
    let mut a: Slice<Array<i64, 2>> = Slice::uninitialized(5);
    for i in 0..5 {
        for j in 0..2 {
            a.get_mut(i).unwrap().set(j, &((i * 2 + j) as i64));
        }
    }
\end{lstlisting}
To modify elements in a simple \verb|Array| or \verb|Slice| use.
\begin{lstlisting}
    let mut a: Array<ClearModp, 10> = Array::uninitialized();
    a.set(2, &ClearModp::from(1));
    a.set(3, &ClearModp::from(4));
\end{lstlisting}
Now suppose you have a \verb|Slice| of \verb|Array|s
\begin{lstlisting}
    let mut S: Slice<Array<ClearModp, 2>> = Slice::uninitialized(5);
\end{lstlisting}
and after some processing you would like to take the fourth element of the
\verb|Slice|.
\begin{lstlisting}
    let mut A = *S.get_mut(4).unwrap();
\end{lstlisting}
The value \verb|A| is now an \verb|Array| of length two.
What you really want is for \verb|A| to be a \verb|Copy| of \verb|S|.
But not all types in Rust enable copying, whilst our basic types do
the \verb|Array| and \verb|Slice| types do not.
Thus in C++ terms in the above code the \verb|A| value is really just a `pointer' 
to the fourth \verb|Array| in the \verb|Slice| \verb|S|.
Thus the effect would be that if you changed elements in \verb|A| then you 
also change the elements in \verb|S|.
In addition if \verb|S| goes out of scope and gets deleted then so will \verb|A|.

To avoid this problem you need to \verb|clone| the output of the \verb|get| as in
\begin{lstlisting}
    let mut A = S.get(4).unwrap().clone();
\end{lstlisting}
But you only need to do this as the \verb|Array| type is not copyable.
If you had the following 
\begin{lstlisting}
    let mut A : Array<ClearModp, 10> = Array::uninitialized();
    a.set(3, ClearModp::from(4));
    let mut a = *A.get(3).unwrap();
    a = a + 3;
\end{lstlisting}
then \verb|a| really is a copy of the entry in \verb|A|. So
at the end we have $a=7$ and $A[3]=4$.

The \verb|unchecked| versions of the \verb|get| operations should only be
used if you know what you are doing. However, the checked versions do have
a performance cost; they require a run-time branch which may impact the
optimizers ability to reduce the total number of rounds of communication.

For more details on \verb|Option|s see
\begin{itemize}
        \item \url{https://doc.rust-lang.org/std/option/}
\end{itemize}





