\mainsection{Control Flow}
In this section we detail the control flow operations which we
support.

\subsection{If-then-else}
Branching on clear data is possible via the standard
rust if-then-else construct.
If $c$ is a $\ints$ variable, then to branch on $c=1$
you need to do
\begin{lstlisting}
  if c == 1 {
	  ...
  }
\end{lstlisting}
In contrast to Mamba, if you need to access data which has been 
altered within the if clauses you do not need to store these in 
memory.
Instead you can just use the standard Rust local variables
and everything will work as expected.
For example
\begin{lstlisting}
    let c0 = black_box(0);

    // Testing c0 = False
    let a = if c0 == 1 {
        ClearModp::from(ConstI32::<0>)
    } else {
        ClearModp::from(ConstI32::<25>)
    };
    // use `a` from here on
\end{lstlisting}

\subsection{While Loops}
While loops can also be executed over $\ints$ datatypes, for
example
\begin{lstlisting}
    let mut cond: i64 = 0;
    while cond==0 {
        ... Do Something ...
        cond = ... some condition ...
    }
\end{lstlisting}
If you want to access something inside the loop that can still be
used outside the loop afterwards, use local variables similar to
how the $cond$ variable is used.

\subsection{For Loops}
Again standard rust loops can be executed, for example
\begin{lstlisting}
    let mut res = 1;
    for i in 2..n  {
        res *= i;
    }
    a=a+res;
\end{lstlisting}
The compiler decides on heuristic optimization rules how and whether
to unroll such loops. Whether the loop is unrolled or not has no
effect on the correctness of the loop, but it can have effects on
the efficiency of cryptographic operations.

\subsection{Function Calls}
Function calls can either be placed inline (which is really only a
    {\em hint} to the compiler) or they can be performed using
stack-pushing of arguments, call/return and then popping outputs.
The former corresponds to the Mamba `Python-Function' behaviour,
whilst the latter corresponds to the Mamba `Mamba-Function' behaviour.
However, with Rust if the compiler decides that the expanded version
is too big it will revert to the call/return variant.

In the following \verb|foo| is an unrolled function,
whilst \verb|bar| is a call/return style function.

\begin{lstlisting}
    #[inline(always)]
    fn foo(x: i64) -> i64 {
        let mut y=1;
        for i in 2..x
           y *= i;
        y
    }

    #[inline(never)]
    fn bar(x: i64) -> i64 {
        let mut y=1;
        for i in 2..x
           y *= i;
        y
    }



    #[inline(never)]
    fn main() {
        let x = foo(10);
        let y = bar(10);
        ...
    }
\end{lstlisting}
When using call/return the compiler works out all the stack pushing
and popping for you.
This is why usage of the stacks by the user might be unsafe, as the
compiler really needs direct access to the stacks.

Function calls can also be recursive, unlike function calls in Mamba,
\begin{lstlisting}
    #[inline(never)]
    fn fibonacci(x: u64) -> u64 {
        print!(x as i64, "\n");
        match x {
            0 => 0,
            1 | 2 => 1,
            _ => fibonacci(x - 1) + fibonacci(x - 2)
        }
    }
\end{lstlisting}
Note that the compiler is much less likely to fully inline recursive
functions unless it can tell when the recursion will end.
