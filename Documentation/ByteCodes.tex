
\mainsection{Run Time Subsystem}
\label{ch:vm}

This chapter describes the implementation of the
MPC Virtual Machine. 
The virtual machine is driven by Bytecodes which
are produced by the MAMBA compiler (see later).
Of course you could compile Bytecodes from any compiler
if you wanted to write one with the correct backend.

The virtual machine structure resembles that of a
simple multi-core processor, and is a register-based
machine.
Each core corresponds to a seperate online thread
of execution; so from now on we refer to these ``cores''
as threads.
Each thread has a seperate set of registers, as
well as a stack for {\em integer} values.
To allow the saving of state, or the transfer of data
between threads there is a global memory.
This global memory (or at least the first $2^{20}$
values) are saved whenever the SCALE system gracefully
shuts down.
The loading of this saved memory into a future run of the
system is controlled by the command line arguments
passed to the \verb+Player.x+ program.
The design is deliberately kept sparse to ensure a fast, low-level 
implementation, whilst more complex optimization decisions are intended 
to be handled at a higher level.

\subsection{Overview}

The core of the virtual machine is a set of threads, which
execute sequences of instructions encoded in a bytecode format.
Files of bytecode instructions are referred to as \emph{tapes},
and each thread processes a single tape at a time.
Each of these execution threads has a pairwise point-to-point
communication channels with the associated threads in all
other players' runtime environments.
These communication channels, like all communication channels
in SCALE, are secured via TLS.
The threads actually have two channels to the correspond
thread in the other parties; we call these different channels
``connections''.
For the online thread ``connection'' zero is used for standard
opening of shared data, whereas ``connection'' one is used for
private input and output. This is to avoid conflicts beween
the two (for example a \verb+PRIVATE_OUTPUT+ coming between
a \verb+STARTOPEN+ and a \verb+STOPOPEN+.
Each online thread is supported by four other threads
performing the offline phase, each again with pairwise 
TLS secured point-to-point channels. Currently the offline
threads only communicate on ``connection'' zero.

In the case of Full Threshold secret sharing another set of
threads act as a factory for FHE ciphertexts. Actively secure
production of such ciphertexts is expensive, requiring complex
zero-knowledge proofs (see Section \ref{sec:fhe}). Thus
the FHE-Factory threads locates this production into a
single location. The number of FHE Factory threads can be
controlled at run-time by the user.

In addition to bytecode files, each program to be run must
have a \emph{schedule}. This is a file detailing the execution
order of the tapes, and which tapes are to be run in parallel.
There is no limit to the number of concurrent tapes specified in a schedule, 
but in practice one will be restricted by the number of cores.
The schedule file allows you to schedule concurrent threads
of execution, it also defines the maximum number of threads
a given run-time system will support. It also defines
the specific bytecode sequences which are pre-loaded
into the system.
One can also programmatically control execution of new
threads using the byte-code instructions \verb+RUN_TAPE+ and \verb+JOIN_TAPE+
(see below for details).
The schedule is run by the \emph{control thread}.
This thread takes the tapes to be executed at a
given step in the schedule, passes them to the execution
threads, and waits for the threads to
finish their execution before proceeding to the next stage of
the schedule.


Communication between threads is handled by a global
\emph{main memory}, which all threads have access to.
To avoid unnecessary stalls there is no locking mechanism provided to
the memory. So if two simultaneously running threads
execute a read and a write, or two writes, to the same
memory location then the result is undefined since it is
not specified as to which order the instructions
will be performed in.
Memory comes in both clear and secret forms.


Each execution thread also has its own local clear and secret
registers, to hold temporary variables.
To avoid confusion with the main memory
The values of registers are not assumed to be maintained
between an execution thread running one tape and
the next tape, so all passing of values
between two sequential tape executions must be done
by reading and writing to the virtual machine's main memory.
This holds even if the two consequtive bytecode
sequences run  on the same ``core''.

\subsection{Bytecode instructions}

The design of the Bytecode instructions within a tape are influenced
by the RISC design strategy, coming in only a few basic
types and mostly taking between one and three
operands. The virtual machine also supports a limited form of
SIMD instructions within a thread, whereby a single instruction is used to
perform the same operation on a fixed size set of registers.
These vectorized instructions are not executed in parallel
as in traditional SIMD architectures, but exist to provide a compact way of
executing multiple instructions within a thread, saving on memory
and code size.

A complete set of byte codes and descriptions is
given in the html file in 
\begin{center}
   \verb+$(HOME)/Documentation/Compiler_Documentation/index.html+
\end{center}
under the class \verb+instructions+.
Each encoded instruction begins with 32 bits reserved for the opcode.
The right-most nine bits specify the instruction to be executed\footnote{The choice of nine is to enable extension of the system later, as eight is probably going 
to be too small.}.
The remaining 23 bits are used for vector instructions, specifying the
size of the vector of registers being operated on.
The remainder of an instruction encoding consists of 4-byte operands, which
correspond to either indices of registers or immediate integer values.
\begin{itemize}
\item Note, vector instructions are not listed in the \verb+html+ document above.
They have the same name as standard instructions, prefixed by `V',
with the opcode created as described above.
\end{itemize}
The basic syntax used in the above html file is as follows:
\begin{itemize}
\item `c[w]': clear register, with the optional suffix `w' if the register is
written to.
\item `s[w]': secret register, as above.
\item `r[w]': regint register, as above.
\item `i'   : 32-bit integer signed immediate value.
\item `int' : 64-bit integer unsigned immediate value.
\item `p'   : 32-bit number representing a player index.
\item `str' : A four byte string.
\end{itemize}
Memory comes in three varieties \verb+sint+, \verb+cint+, and
\verb+regint+; denoted by \verb+S[i]+, \verb+C[i]+ and \verb+R[i]+.

\subsection{Load, Store and Memory Instructions}
Being a RISC design the main operations are load/store
operations, moving operations, and memory operations.
Each type of instructions comes in either clear data, 
sharedata, or integer data formats. 
The integer data is pure integer arithemetic, say
for controling loops, whereas clear data is integer
arithmetic modulo $p$.
For the clear values all values represented as integers 
in the range $(-\frac{p-1}{2}, \dots, \frac{p-1}{2}]$.
The integer stack is again mainly intended for loop
control.

\subsubsection{Basic Load/Store/Move Instructions:}
\verb+LDI+,
\verb+LDI+,
\verb+LDSI+,
\verb+LDINT+,
\verb+MOVC+,
\verb+MOVS+,
\verb+MOVINT+.

\subsubsection{Loading to/from Memory:}
 \verb+LDMC+,
 \verb+LDMS+,
 \verb+STMC+,
 \verb+STMS+,
 \verb+LDMCI+,
 \verb+LDMSI+,
 \verb+STMCI+,
 \verb+STMSI+,
 \verb+LDMINT+, 
 \verb+STMINT+,
 \verb+LDMINTI+,
 \verb+STMINTI+.

\subsubsection{Accessing the integer stack:}
\verb+PUSHINT+, \verb+POPINT+.

\subsubsection{Data Conversion}
To convert from mod $p$ to integer values and
back we provide the conversion routines.
\verb+CONVINT+, \verb+CONVMODP+.
These are needed as the internal mod $p$ representation
of clear data is in Montgomery representation.

\subsection{Preprocessing loading instructions}
The instructions for loading data from the preprocessing phase
are denoted \verb+TRIPLE+, \verb+SQUARE+, \verb+BIT+,
and they take as argument three, two, and one secret registers 
respectively.
The associated data is loaded from the concurrently running
offline threads and loaded into the registers given as arguments.

\subsection{Open instructions}

The process of opening secret values is covered by two instructions.
The \verb+STARTOPEN+ instruction takes as input a set of $m$
shared registers, and \verb+STOPOPEN+ an associated set of $m$
clear registers, where $m$ can be an arbitrary integer.
This initiates the protocol to reveal the $m$ secret shared register values,
storing the result in the specified clear registers. The reason for
splitting this into two instructions is so that local, independent
operations may be placed between a \verb+STARTOPEN+ and \verb+STOPOPEN+,
to be executed whilst waiting for the communication to finish.

There is no limit on the number of operands to these instructions,
allowing for communication to be batched into a single pair of
instructions to save on network latency. However, note that when
the \texttt{RunOpenCheck} function in the C++ class \texttt{Open\_Protocol}
is used to check MACs/Hashes then this can stall when the network buffer fills 
up, and hang indefinitely.
On our test machines this happens when opening around 10000 elements
at once, so care must be taken to avoid this when compiling or writing
bytecode (the Python compiler could automatically detect and avoid
this).

\subsection{Threading tools}
Various special instructions are provided to ease the workload when writing
programs that use multiple tapes. 
\begin{itemize}
\item The \verb+LDTN+ instruction loads the current thread number into 
a clear register.
\item The \verb+LDARG+ instruction loads an argument that was passed 
when the current thread was called.
Thread arguments are optional and consist of a single integer, 
which is specified in the schedule file that determines the execution 
order of tapes, or via the instruction \verb+RUN_TAPE+.
\item The \verb+STARG+ allows the current tape to change its
existing argument.
\item To run a specified pre-loaded tape in a given thread, with
a given argument the \verb+RUN_TAPE+ command is executed.
\item To wait until a specified thread has finished one executes
the \verb+JOIN_TAPE+ function.
\end{itemize}

\subsection{Basic Arithmetic}
This is captured by the following instructions,
with different instructions being able to be operated
on clear, shared and integer types.
    \verb+ADDC+,
    \verb+ADDS+,
    \verb+ADDM+,
    \verb+ADDCI+,
    \verb+ADDSI+,
    \verb+ADDINT+,
    \verb+SUBC+,
    \verb+SUBS+,
    \verb+SUBML+,
    \verb+SUBMR+,
    \verb+SUBCI+,
    \verb+SUBSI+,
    \verb+SUBCFI+,
    \verb+SUBSFI+,
    \verb+SUBINT+,
    \verb+MULC+,
    \verb+MULM+,
    \verb+MULCI+,
    \verb+MULSI+,
and
    \verb+MULINT+.

\subsection{Advanced Arithmetic}
More elaborate algorithms can clearly be executed directly on
clear or integer values; without the need for complex
protocols. These include logical, shift and number
theoretic functions.
    \verb+ANDC+,
    \verb+XORC+,
    \verb+ORC+,
    \verb+ANDCI+,
    \verb+XORCI+,
    \verb+ORCI+,
    \verb+NOTC+,
    \verb+SHLC+,
    \verb+SHRC+,
    \verb+SHLCI+,
    \verb+SHRCI+,
    \verb+DIVC+,
    \verb+DIVCI+,
    \verb+DIVINT+,
    \verb+MODC+,
    \verb+MODCI+.
    \verb+LEGENDREC+,
and
    \verb+DIGESTC+.

\subsection{Debuging Output}
To enable debugging we provide simple commands to send
debugging information to the \verb+Input_Output+ class.
These bytecodes are
\begin{verbatim}
    PRINTINT,           PRINTMEM,          PRINTREG,           PRINTREGPLAIN,
    PRINTCHR,           PRINTSTR,          PRINTCHRINT,        PRINTSTRINT,
    PRINTFLOATPLAIN,    PRINTFIXPLAIN.
\end{verbatim}

\subsection{Data input and output}
This is entirely dealt with in the later Chapter on IO.
The associated bytecodes are
\begin{verbatim}
             OUTPUT_CLEAR,           INPUT_CLEAR, 
             OUTPUT_SHARE,           INPUT_SHARE, 
             OUTPUT_INT,             INPUT_INT,
             PRIVATE_INPUT,          PRIVATE_OUTPUT,
             OPEN_CHAN,              CLOSE_CHAN
\end{verbatim}

\subsection{Branching}
Branching is supported by the following
instructions
 \verb+JMP+,
    \verb+JMPNZ+,
    \verb+JMPEQZ+,
    \verb+EQZINT+,
    \verb+LTZINT+,
    \verb+LTINT+,
    \verb+GTINT+,
    \verb+EQINT+,
and
    \verb+JMPI+.

\subsection{Other Commands}
The following byte codes are for fine tuning the machine
\begin{itemize}
\item \verb+REQBL+ this is output by the compiler to signal that
the tape requires a minimal bit length. This forces the runtime
to check the prime $p$ satisfies this constraint.
\item \verb+CRASH+ this enables the program to create a crash,
if the programmer is feeling particularly destuctive.
\item \verb+RAND+ this loads a pseudo-random value into a
clear register. This is not a true random number, as all
parties output the same random number at this point.
\item \verb+RESTART+ which restarts the online runtime.
See Section \ref{sec:restart} for how this intended to be used.
\item \verb+CLEAR_MEMORY+ which clears the current memory.
See Section \ref{sec:restart} for more details on how this is used.
\item \verb+CLEAR_REGISTERS+ which clears the registers of this processor core (i.e. thread).
See Section \ref{sec:restart} for more details on how this is used.
\item \verb+START_TIMER+ and \verb+STOP_TIMER+ are used to time different
parts of the code. There are 100 times available in the system;
each is initialized to zero at the start of the machine running.
The operation \verb+START_TIMER+ re-initializes a specified timer,
whereas \verb+STOP_TIMER+ prints the elapsed time since the last
initialization (it does not actually reinitialise/stop the timer itself).
These are accessed from MAMBA via the functions
\verb+start_timer(n)+ and \verb+stop_timer(n)+.
The timers use an internal class to measure time command in the C runtime. 
\end{itemize}



