\mainsection{Advanced Protocols}
This section details the protocols needed to perform complex
functionalities on LSSS style secret shared values in MAMBA.
As well as the documentation here further, details can be
found at
\begin{center}
  \verb+$(HOME)/Documentation/Compiler_Documentation/index.html+
\end{center}
under the heading \verb+Files+.

The protocols break the ``arithmetic circuit'' model of computation,
as they make heavy use of pre-processed data and the ability
to $\Open$ shared values.
In particular we assume three lists of pre-processed data:
\[ \MList, \quad \SList, \quad \BList. \]
An entry on the $\MList$ is of the form
$(\sshare{a}, \sshare{b}, \sshare{c})$ where $c=a \cdot b \pmod{p}$,
an entry on the $\SList$ is of the form
$(\sshare{a}, \sshare{b})$ where $b=a^2 \pmod{p}$,
whilst an entry on the $\BList$ is of the form
$\sshare{b}$ where $b \in \{0,1\}$.
We also assume a function $\Random()$ which can generate
a random secret sharing (this can be implemented by just taking
the first element from a multiplication triple).
We add and multiply secret shared elements in what follows
using the notation
\[ \sshare{a+b} \asn \sshare{a}+\sshare{b}, \quad
  \sshare{a\cdot b} \asn \sshare{a} \cdot \sshare{b}.
\]

The protocols in this chapter allow us to do more advanced operations,
without resorting to using full blown arithmetic circuits.
We describe here what we have implemented as a form of
documentation, both for us and for others.
Many of the protocol ideas can be found in the five
documents
\begin{itemize}
  \item Unconditionally Secure Constant-Rounds
        Multi-party Computation for Equality, Comparison, Bits and Exponentiation.
          {\em TCC 2006}, \cite{DFKNT06}.
  \item Improved Primitives for Secure Multiparty Integer Computation.
          {\em SCN 2010}, \cite{CH10}.
  \item D9.2 of the EU project {\em secureSCM}, \cite{SSCM}.
  \item Secure Computation with Fixed-Point Numbers {\em FC 2010} \cite{CS10}.
  \item Secure Computation on Floating Point Numbers {\em NDSS 2013} \cite{ABZS13}.
\end{itemize}
Many of the protocols operating on integers make use of a statistical security
parameter $\kappa$.
If the integer being operated on is $k$ bits long, then we often require $(k+\kappa)<\log_2 p$.
For ease of implementation of the protocols we recommend $k$ is
always a power of two, and we assume this in the write up below.
If this is not the case, obvious tweaks can be made to the protocols.

Due to experience with the SPDZ system we prefer logarithmic
round protocols over constant round protocols, it appears that in
practice the logarithmic round protocols outperform the constant
round ones.
The MAMBA compiler can execute non-constant round protocols, by
flicking a compile time switch. But here we only document
logarithmic round protocols.

\msubsection{Basic Protocols}

\msubsubsection{$\mathsf{Inv}(\sshare{x})$:}
This produces a share $\sshare{z}$ of $1/x \pmod{p}$, with an
$\abort$ if $x=0$.
\begin{enumerate}
  \item $\sshare{a} \asn \Random()$.
  \item $\sshare{y} \asn \sshare{a} \cdot \sshare{x}$.
  \item $y \asn \Open(\sshare{y})$.
  \item If $y=0$ then $\abort$.
  \item $t \asn 1/y \pmod{p}$.
  \item $\sshare{z} \asn t \cdot \sshare{a}$.
  \item Return $\sshare{z}$.
\end{enumerate}
\paragraph{MAMBA Example:} To obtain the inverse of a \verb|sint| or a \verb|cint| can be done as follows:
\begin{lstlisting}[language={python}]
d = sint(5)
d_inv =  AdvInteger.Inv(d)
print_ln("inverse is correct if 1: %s", (d*d_inv).reveal())
\end{lstlisting}

\msubsubsection{$\mathsf{Ran}_p^*()$:}
This produces a random sharing of a value $x$ and its inverse $1/x$.
This is faster than generating $x$, and then performing the above operaton.
\begin{enumerate}
  \item Take a triple $(\sshare{a},\sshare{b},\sshare{c})$ from $\MList$.
  \item $c \asn \Open(\sshare{c})$.
  \item If $c=0$ then return to the first step.
  \item $\sshare{a^{-1}} \asn c^{-1} \cdot \sshare{b}$.
  \item Output $(\sshare{a},\sshare{a^{-1}})$.
\end{enumerate}
This function does not exist ``as is'' in the MAMBA language, as it
is used in place within the python compiler. Thus it is only here
for documentation reasons.

\msubsubsection{$\mathsf{PreMult}(\sshare{a_1},\ldots,\sshare{a_t},T)$:}
This computes the prefix multiplication, i.e. the values
\[ \sshare{a_{i_0,i_1}} = \bsshare{\prod_{i=i_0}^{i_1} a_i} \]
where $(i_0,i_1) \in T$ and $1 \le i_0 \le i_1 \le t$.
\begin{enumerate}
  \item For $i \in [0,\ldots,t]$ do.
        \begin{enumerate}
          \item $(\sshare{b_i},\sshare{b_i^{-1}}) \asn \mathsf{Ran}_p^*()$.
        \end{enumerate}
  \item For $i \in [0,\ldots,t]$ do.
        \begin{enumerate}
          \item $\sshare{t} \asn \sshare{b_{i-1}} \cdot \sshare{a_i}$.
          \item $\sshare{d_i} \asn \sshare{t} \cdot \sshare{b_i^{-1}}$.
          \item $d_i \asn \Open(\sshare{d_i})$.
        \end{enumerate}
  \item For $(i_0,i_1) \in T$
        \begin{enumerate}
          \item $d_{i_0,i_1} \asn \prod_{i=i_0}^{i_1} d_i$.
          \item $\sshare{a_{i_0,i_1}} \asn d_{i_0,i_1} \cdot
                  \sshare{b_{i_0-1}^{-1}} \cdot \sshare{b_{i_1}}$.
        \end{enumerate}
\end{enumerate}
Again, this function does not exist ``as is'' in the MAMBA language, as it
is used in place within the python compiler. Thus it is only here
for documentation reasons.



\msubsection{Bit Oriented Operations}
These operations refer exclusively to how to perform bit oriented operations on top of
$\modp$ data-types. It is not related to the \verb|sbit| type.

\msubsubsection{$\mathsf{OR}(\sshare{a},\sshare{b})$:}
This computes the logical OR of two input shared bits:
\begin{enumerate}
  \item Return $\sshare{a}+\sshare{b}-\sshare{a} \cdot \sshare{b}$.
\end{enumerate}
\paragraph{MAMBA Example:} To obtain the \verb|or| of two \verb|sint| or \verb|cint| numbers you can:
\begin{lstlisting}[language={python}]
a= sint(1)
b= sint(0)
print_ln("or is correct if 1: %s", (a or b).reveal())
print_ln("or is correct if 1: %s", AdvInteger.or_op(a, b).reveal())
\end{lstlisting}

\msubsubsection{$\mathsf{XOR}(\sshare{a},\sshare{b})$:}
This computes the logical XOR of two input shared bits:
\begin{enumerate}
  \item Return $\sshare{a}+\sshare{b}-2 \cdot \sshare{a} \cdot \sshare{b}$.
\end{enumerate}
\paragraph{MAMBA Example:} To obtain the \verb|xor| of two \verb|sint| or \verb|cint| numbers you can:
\begin{lstlisting}[language={python}]
a= sint(1)
b= sint(0)
print_ln("or is correct if 1: %s", AdvInteger.xor_op(a, b).reveal())
\end{lstlisting}

\iffalse
  \note{Nigel}{Does not seem to be used anywhere}
  \msubsubsection{$\mathsf{Symm}(f,\sshare{a_1},\ldots,\sshare{a_t})$:}
  This takes a symmetric boolean function $f$ on $t$ binary inputs
  and evaluates it at the points $\sshare{a_1},\ldots,\sshare{a_t}$,
  where we assume $a_i \in \{0,1\}$.
  We first pre-process the function $f$ so that we can write
  \[ f(x_1,\ldots,x_t) = \phi(1+\sum_{i=1}^t x_i), \]
  where the inner sum is over the integers and
  $\phi:{1,2,\ldots,t+1} \longrightarrow \{0,1\}$.
  We then write
  \[ \phi(X) = \sum_{i=0}^t \alpha_i \cdot X^i \pmod{p} \]
  using Lagrange interpolation.
  So for example if we have $f(X_1,X_2)=X_1^2+X_1 \cdot X_2+X_2^2+1$
  then $\phi$ is the function which maps
  $1 \longrightarrow 1$, $2 \longrightarrow 0$, $3 \longrightarrow 0$.
  In which case $\phi(X)=X^2/2-5 \cdot X/2+3$.

  To compute this function we have the algorithm
  \begin{enumerate}
    \item $\sshare{a} \asn 1+ \sum_{i=1}^t \sshare{a_i}$.
    \item $(\sshare{a},\sshare{a^2},\ldots,\sshare{a^t})
            \asn \mathsf{PreMult}(\sshare{a},\ldots,\sshare{a},
            \{(1,1),\ldots,(1,t)\})$.
    \item $\sshare{f(a_1,\ldots,a_t)}
            \asn \sum_{i=0}^t \alpha_i \cdot \sshare{a^i}$.
  \end{enumerate}
\fi

\msubsubsection{$\mathsf{KOp}(\odot,\sshare{a_1},\ldots,\sshare{a_k},k)$:}
This computes the operation $\sshare{p}= \bigodot_{i=1}^k \sshare{a_i}$
given a binary operator $\odot$.
\begin{enumerate}
  \item If $k>1$ then (assuming $k$ is even, if not pad)
        \begin{enumerate}
          \item $\sshare{t_1} \asn \mathsf{KOp}(\odot, \sshare{a_{k/2}},\ldots,\sshare{a_{k/2}},k/2)$.
          \item $\sshare{t_2} \asn \mathsf{KOp}(\odot, \sshare{a_{k/2+1}},\ldots,\sshare{a_k},k/2)$.
          \item Return $\sshare{t_1} \odot \sshare{t_2}$.
        \end{enumerate}
  \item Else
        \begin{enumerate}
          \item $\sshare{p} \asn \sshare{a_1}$.
        \end{enumerate}
  \item Return $\sshare{p}$.
\end{enumerate}

\paragraph{MAMBA Example:} Note that we basically want to achieve a construction capable to call any function in an iterative fashion reducing computation time. In this sense a call to the function could be perform in the following way:
\begin{lstlisting}[language={python}]
# addition
def addition(a, b):
    return a + b

ar=[1]*16
# (k is exctracted from ar directly on the implementation)
print_ln("KOpL is correct if 32: %s", (AdvInteger.KOpL(addition,ar)).reveal())
\end{lstlisting}
This runs in logarithmic rounds.

\msubsubsection{$\mathsf{PreOp}(\odot,\sshare{a_1},\ldots,\sshare{a_k},k)$:}
This computes the prefix operator $\sshare{p_j} = \odot_{i=1}^j \sshare{a_i}$,
for $1 \le j \le k$.
\begin{enumerate}
  \item For $i \in [1,\ldots, \log_2 k]$ do
        \begin{enumerate}
          \item For $j \in [1,\ldots,k/2^i]$ do
                \begin{enumerate}
                  \item $y \asn 2^{i-1}+j\cdot 2^i$.
                  \item For $z \in [1,\ldots,2^{i-1}]$ do
                        \begin{enumerate}
                          \item $\sshare{a_{y+z}} \asn \sshare{a_y} \odot \sshare{a_{y+z}}$.
                        \end{enumerate}
                \end{enumerate}
        \end{enumerate}
  \item Return $(\sshare{a_1},\ldots,\sshare{a_k})$.
\end{enumerate}

\paragraph{MAMBA Example:} Similarly, we basically want to achieve a construction capable to call any function in an iterative fashion reducing computation time. In this case, however, we return all the list of intermediate values. The function call could be performed in the following way:
\begin{lstlisting}[language={python}]
def addition(a, b):
    return a + b

def addition_triple(a, b, c):
	# c is a boolean parameter
    return a + b

e = sint(2)

ar = [e]*16
# k is stracted from ar.
print_ln("PreOpL is correct if 32: %s", (AdvInteger.PreOpL(addition_triple,ar))[15].reveal())
print_ln("PreOpN is correct if 32: %s", (AdvInteger.PreOpN(addition,ar))[15].reveal())
\end{lstlisting}
Note that both methods are implementations of the functionality with slightly different  communication and round complexity.
With the \verb+PreOpL+ function corresponding to the pseudo-code above.

\msubsubsection{$\mathsf{Sum\mhyphen Bits}(\sshare{x}_B)$:}
This outputs a shared integer $x$ in the range
$[0,\ldots,2^k)$ with $2^k<p$ given the input $k$ bits
$\sshare{x}_B$ making up its binary representation.
\begin{enumerate}
  \item $\sshare{x} \asn \sum_{i=0}^{k-1} 2^i \cdot \sshare{x_i}$.
  \item Output $\sshare{x}$.
\end{enumerate}
To ease notation in what follows we will write
$\sshare{x}_B = \{\sshare{x_i}\}_{i=0}^{k-1}$.

\paragraph{MAMBA Example:}  We can reconstruct a number from its bits as follows:
\begin{lstlisting}[language={python}]
a = [sint(0)]*program.bit_length
# k is taken from a
print_ln("sum_bits is correct if 0: %s", AdvInteger.SumBits(a, program.bit_length).reveal())
\end{lstlisting}

\msubsubsection{$\mathsf{PRandInt}(k)$:}
This generates a random secret integer $r$ in the range $[0,\ldots,2^k-1]$.
In the pseudo-code below we let $[\BList]\star k$ denote taking
$k$ bits from $\BList$.
\begin{enumerate}
  \item $\sshare{r}_B \asn [\BList]\star k$.
  \item $\sshare{r} \asn \mathsf{Sum\mhyphen Bits}(\sshare{r}_B)$.
  \item Return $\sshare{r}$.
\end{enumerate}
\paragraph{MAMBA Example:}  We obtain the value as follows:
\begin{lstlisting}[language={python}]
# x is the returned value
x = sint()

# k are parameters
k = 5
AdvInteger.PRandInt(x, k)
\end{lstlisting}

\msubsubsection{$\mathsf{PRandM}(k,m,\kappa)$:}
This generates two random shares $r' \in [0,\ldots,2^{k+\kappa-m}-1]$
and $r \in [0,\ldots,2^m-1]$, along with the shares the bits of $r$.
\begin{enumerate}
  \item $\sshare{r}_B \asn [\BList]\star m$.
  \item $\sshare{r} \asn \mathsf{Sum\mhyphen Bits}(\sshare{r}_B)$.
  \item $\sshare{r'} \asn \mathsf{PRandInt}(k+\kappa-m)$.
  \item Return $\sshare{r'}, \sshare{r}, \sshare{r}_B$.
\end{enumerate}
\paragraph{MAMBA Example:}  We obtain the randomness and its bits as follows:
\begin{lstlisting}[language={python}]

# x, y,z are returned values
x = sint()
y = sint()
yB = [sint() for i in range(3)]

# k, m, kappa are parameters
k = 5
m = 3
kappa = 7
AdvInteger.PRandM(x, y, yB, k, m, kappa)
\end{lstlisting}

\msubsubsection{$\mathsf{CarryOut}(\sshare{a}_B,\sshare{b}_B,k)$:}
This protocol computes the carry-out of a binary addition of
two $k$ bit shared values, when presented via shared bits.
The protocol can easily be adapted to the case when either
the bits of $a$, or the bits of $b$, are given in the clear.
We give a logarithmic round version, which requires $\log k$
rounds of interaction.
It requires a sub-routine $\mathsf{CarryOutAux}$ which we
give below.
\begin{enumerate}
  \item For $i \in [0,\ldots,k-1]$ do
        \begin{enumerate}
          \item $\sshare{d_i}_B \asn (\mathsf{XOR}(\sshare{a_i},\sshare{b_i}),
                  \sshare{a_i}\cdot \sshare{b_i})$.
                  [Note, $\sshare{d_i}_B$ is a set of two shared bits, one being the XOR
                    of $a_i$ and $b_i$, whilst the other the AND].
        \end{enumerate}
  \item $\sshare{d}_B \asn \mathsf{CarryOutAux}(\sshare{d_{k-1}}_B,\ldots,\sshare{d_0}_B,k)$.
  \item $(\sshare{p},\sshare{g}) \asn \sshare{d}_B$.
  \item Return $\sshare{g}$.
\end{enumerate}
\paragraph{MAMBA Example:}  The carry out operation is executed as follows:
\begin{lstlisting}[language={python}]
res = sint() # last carry bit in addition of a and b
a = [cint(i) for i in [1,0]] # array of clear bits
b = [sint(i) for i in [0,1]] # array of secret bits (same length as a)
c = 0 # initial carry-in bit
AdvInteger.CarryOut(res, a, b, c)
\end{lstlisting}

\msubsubsection{$\mathsf{CarryOutAux}(\sshare{d_k}_B,\ldots,\sshare{d_1}_B,k)$:}
This function uses the $\circ$ operator for carry propagation on two
bit inputs which is defined as
\[  \circ:  \left\{ \begin{array}{ccc}
    \{0,1\}^2 \times \{0,1\}^2 & \longrightarrow & \{0,1\} \\
    (p_2,g_2) \circ (p_1,g_1)  & \longmapsto     &
    (p_1 \wedge p_2, g_2 \vee (p_2 \wedge g_1))
  \end{array} \right.
\]
This is computed using arithmetic operations (i.e. where the values are
held as bits modulo $p$) as $(p,g) = (p_2,g_2) \circ (p_1,g_1)$ via
\begin{align*}
  p & = p_1 \cdot p_2,       \\
  g & = g_2 + p_2 \cdot g_1.
\end{align*}
Given this operation the function $\mathsf{CarryOutAux}$ is defined by,
which is just a specialisation of the protocol $\mathsf{KOp}$ above,
\begin{enumerate}
  \item If $k>1$ then
        \begin{enumerate}
          \item For $i \in [1,\ldots,k/2]$ do
                \begin{enumerate}
                  \item $\sshare{u_i}_B \asn \sshare{d_{2\cdot i}}_B \circ \sshare{d_{2 \cdot i-1}}_B$.
                \end{enumerate}
          \item $\sshare{d}_B \asn \mathsf{CarryOutAux}(\sshare{u_{k/2}}_B,\ldots,\sshare{u_1}_B,k/2)$.
        \end{enumerate}
  \item Else
        \begin{enumerate}
          \item $\sshare{d}_B \asn \sshare{d_1}_B$.
        \end{enumerate}
  \item Return $\sshare{d}_B$.
\end{enumerate}
\paragraph{MAMBA Example:} This method is thought as a subroutine for \verb|CarryOut| and should not be used outside that context. the code is invoked in the following way:
\begin{lstlisting}[language={python}]
kappa = 16
x = [cint(i) for i in [1,0]]
res = sint()
AdvInteger.CarryOutAux(res, x)
\end{lstlisting}


\msubsubsection{$\mathsf{BitAdd}((\sshare{a_{k-1}},\ldots,\sshare{a_0}), (\sshare{b_{k-1}},\ldots,\sshare{b_0}),k)$:}
  This function also makes use of the operator $\circ$.
  The inputs are shared bits. The case where one set of inputs is in the clear
  is obviously more simple, and we do not detail this here.
  \begin{enumerate}
    \item For $i \in [0,\ldots,k-1]$ do
          \begin{enumerate}
            \item $\sshare{d_i}_B \asn (\mathsf{XOR}(\sshare{a_i}, \sshare{b_i}),
                    \sshare{a_i}\cdot \sshare{b_i})$.
          \end{enumerate}
    \item $\sshare{c_{k-1},t_{k-1}},\ldots,\sshare{c_0,t_0}  \asn \mathsf{PreOp}(\circ,\sshare{d_{k-1}}_B,\ldots,\sshare{d_0}_B,k)$.
    \item $\sshare{s_0} \asn \mathsf{XOR}(\sshare{a_0}, \sshare{a_1})$.
    \item For $i \in [1,\ldots,k-1]$ do
          \begin{enumerate}
            \item $\sshare{s_i} \asn \sshare{a_i}+\sshare{b_i}+\sshare{c_{i-1}}-2\cdot \sshare{c_i}$.
          \end{enumerate}
    \item $\sshare{s_k} \asn \sshare{c_{k-1}}$.
    \item Return $(\sshare{s_k},\ldots,\sshare{s_0})$.
  \end{enumerate}
  There is also a variant (requiring less operations) which does an increment
  by one, i.e. $x \asn x+1$.

  \paragraph{MAMBA Example:} The addition of two numbers expressed in bits, can be performed as follows:
  \begin{lstlisting}[language={python}]
a_bits = [sint(i) for i in [0,1,0,1,1]]
b_bits = [sint(i) for i in [0,1,0,1,1]]
# k is extracted from the array size
b = AdvInteger.BitAdd(a_bits, b_bits)
c = AdvInteger.BitIncrement(a_bits)
\end{lstlisting}

  \msubsubsection{$\mathsf{BitLTFull}(\sshare{a}_B,\sshare{b}_B,k)$:}
  This computes the sharing of the bit $a<b$, where both $a$ and $b$ are
  assumed to be $k$ bit values presented in bit format, with
$a=\sum_{i=0}^{k-1} a_i \cdot 2^i$ and
$b=\sum_{i=0}^{k-1} b_i \cdot 2^i$.
  The method used is from \cite{DFKNT06}.
  \paragraph{MAMBA Example:} Comparing two secret shared numbers decomposed in bits can be achieved as follows:
  \begin{lstlisting}[language={python}]
x = [sint(i) for i in [0,1,1]]
y = [sint(i) for i in [1,0,1]]
z = sint()
kappa = 16
# k can be extracted from the array size
# in this case the bit that is the answer is contained in z
AdvInteger.BitLTFull(z, x, y)
\end{lstlisting}

  \msubsubsection{$\mathsf{BitLT}(a,\sshare{b}_B,k)$:}
  This computes the sharing of the bit $a<b$, where $a$ is a public value.
  Both $a$ and $b$ are assumed to be $k$ bit values, with
$a=\sum_{i=0}^{k-1} a_i \cdot 2^i$ and
$b=\sum_{i=0}^{k-1} b_i \cdot 2^i$.
  \begin{enumerate}
    \item For $i \in [0,\ldots,k-1]$
          \begin{enumerate}
            \item $\sshare{b_i'}\asn 1-\sshare{b_i}$.
          \end{enumerate}
    \item $\sshare{s} \asn 1-
            \mathsf{CarryOut}((a_{k-1},\ldots,a_0),\sshare{b'}_B)$.
    \item Return $\sshare{s}$.
  \end{enumerate}
  \paragraph{MAMBA Example:} Comparing an open register with a secret shared number decomposed in bits can be achieved as follows:
  \begin{lstlisting}[language={python}]
x = cint(5)
y = [sint(i) for i in [1,0,1]]
z = sint()
kappa = 16
# k can be extracted from the array size
# in this case the bit that is the answer is contained in z
AdvInteger.BitLT(z, x, y, kappa)
\end{lstlisting}


  \msubsubsection{$\mathsf{BitDecFull}(\sshare{a})$:}
  This produces the bit-decomposition of the shared value $a$ with
  respect to the prime \verb+program.P+.
  The method used is the one from \cite{DBLP:conf/pkc/NishideO07},
  and thus it does not depend on any statistical security gap.
  The next version is more efficient variant, but only
  gives statistical security guarantees, and does not
  work for all $a \in [0,\ldots,p)$ is given.
  When \verb+program.P+ is larger than $64$-bit the
  method from \cite{DFKNT06} is used; not because it is
  faster but purely because SCALE is not so adapt at doing
  the other method for large primes.
  This routine has a side-effect of writing to memory locations in the
  \verb+sint+ memory in the first $0,\ldots,\log_2 p$ locations.
  \paragraph{MAMBA Example:} A secret shared value can be decomposed into bits as shown in the following snippet:
  \begin{lstlisting}[language={python}]
x = cint(23)
bits = AdvInteger.BitDecFull(a)
for i in range(program.P.bit_length()):
  print_str('%s',bits[i].reveal())
\end{lstlisting}


  \msubsubsection{$\mathsf{BitDec}(\sshare{a},k,m, \kappa)$:}
  This outputs the $m$ least significant bits in the $2$'s complement
  representation of $a \in \Zk$.
  It requires a statistical security `gap' of $\kappa$.
  \begin{enumerate}
    \item $\sshare{r'}, \sshare{r}, \sshare{r}_B \asn \mathsf{PRandM}(k,m,\kappa)$.
    \item $c \asn \Open(\sshare{a}+2^k+2^{k+\kappa}-\sshare{r}-2^m \cdot \sshare{r'})$.
    \item $(\sshare{a_{m-1}},\ldots,\sshare{a_0}) \asn
            \mathsf{BitAdd}(c,(\sshare{r_{m-1}},\ldots,\sshare{r_0}))$.
    \item Return $(\sshare{a_{m-1}},\ldots,\sshare{a_0})$.
  \end{enumerate}
  \paragraph{MAMBA Example:} A secret shared value can be decomposed into bits as shown in the following snippet:
  \begin{lstlisting}[language={python}]
a = sint(23)
k = 5
m = 5
kappa = 20
# where b is bit array of type sint
b = AdvInteger.BitDec(a, k, m, kappa)
# you can also vall via (where it selects m=k)
b = a.bit_decompose(k, kappa)
\end{lstlisting}

  \msubsection{Arithmetic with Signed Integers}
  In this section we define basic arithmetic on signed integers.
  We define $\Zk$ as the set of integers $\{x \in \Z: -2^{k-1} \le x \le 2^{k-1}-1\}$,
  which we embed into $\F_p$ via the map $x \mapsto x \pmod{p}$.
  Most algorithms require a statistical security gap of $\kappa$.


  \msubsubsection{$\mathsf{Mod2m}(\sshare{a_{prime}}, \sshare{a},k,m, \kappa, \signed)$:}
  Given $a \in \Zk$, $m \in [1,\ldots,k-1]$ this outputs $a' = a \pmod{2^m}$
  when $\signed=\true$.
  When $\signed=\false$ it can take as input a value in $[0,\ldots,2^k-1]$.
  Use this protocol when $m>1$, for $m=1$ use $\mathsf{Mod2}$ below.
  \begin{enumerate}
    \item $\sshare{r'}, \sshare{r}, \sshare{r}_B \asn \mathsf{PRandM}(k,m,\kappa)$.
    \item $c \asn \Open(\sshare{a}+\signed \cdot 2^{k-1}+\sshare{r}+2^m \cdot \sshare{r'})$.
    \item $c'\asn c \pmod{2^m}$.
    \item $\sshare{u} \asn \mathsf{BitLT}(c',(\sshare{r_{m-1}},\ldots,\sshare{r_0}),m)$.
    \item $\sshare{a'} \asn c'-\sshare{r}+2^m \cdot \sshare{u}$.
    \item Return $\sshare{a'}$.
  \end{enumerate}

  \paragraph{MAMBA Example:} The \verb|mod| to a power of $2$ of a secret shared integer register can be obtain as follows:
  \begin{lstlisting}[language={python}]
a_prime = sint(0) # a % 2 ^ m
a = sint(100)
k = 16 # bit length of a
m = 2 # the power of two, i..e we want  ( a mod 2^m )
kappa = 8
signed = True # True/False

# where a is a register of type sint
r_dprime, r_prime, c, c_prime, u, t, c2k1 = \
    AdvInteger.Mod2m(a_prime, a, k, m, kappa, signed)
\end{lstlisting}


  \msubsubsection{$\mathsf{Mod2}(\sshare{a},k, \kappa, \signed)$:}
  \begin{enumerate}
    \item $\sshare{r'}, \sshare{r}, \sshare{r_0} \asn \mathsf{PRandM}(k,1,\kappa)$.
    \item $c \asn \Open(\sshare{a}+\signed \cdot 2^{k-1}+\sshare{r}+2 \cdot \sshare{r'})$.
    \item $\sshare{a_0} \asn c_0+\sshare{r_0}-2 \cdot c_0 \cdot \sshare{r_0}$.
    \item Return $\sshare{a_0}$.
  \end{enumerate}

  \paragraph{MAMBA Example:} You obtain the modulo two of a number as follows:
  \begin{lstlisting}[language={python}]
a = sint(1)
a_0 = sint()
k = 1
kappa = 8
signed = False
# y stores the result of X % 2
AdvInteger.Mod2(a_0, A, k, kappa, signed)
\end{lstlisting}

  \msubsubsection{Addition, Multiplication in $\Zk$}
  Addition and multiplication $\odot$ of two elements $\sshare{a}, \sshare{b}$
  to obtain $\sshare{c}$, where $a,b,c \in \Zk$ is then easy to define
  by performing
  \begin{enumerate}
    \item $\sshare{d} \asn \sshare{a} \odot \sshare{b}$.
    \item $\sshare{c} \asn \mathsf{Mod2m}(\sshare{d},k',k)$,
          where $k'=k+1$ is $\odot=+$ and $k'=2\cdot k$ if $\odot=\cdot$.
    \item Return $\sshare{c}$.
  \end{enumerate}
  These functions are not directly callable from MAMBA, they are
  included here purely for documentation reasons.

  \msubsubsection{$\mathsf{Pow2}(\sshare{a},k, \kappa)$:}
  This computes $\sshare{2^a}$ where $a \in [0,\ldots,k)$
  \begin{enumerate}
    \item $m \asn \ceil{\log_2 k}$.
    \item $\sshare{a_{m-1}},\ldots,\sshare{a_0} \asn \mathsf{BitDec}(\sshare{a},m,m)$.
    \item For $i\in [0,\ldots,m-1]$ do
          \begin{enumerate}
            \item $\sshare{v_i} \asn 2^{2^i} \cdot \sshare{a_i}+1-\sshare{a_i}$.
          \end{enumerate}
    \item $\sshare{x_0},\ldots,\sshare{x_{m-1}}
            \asn \ \mathsf{PreMult}(\sshare{v_0},\ldots,\sshare{v_{m-1}}, \{(1,1),\ldots,(1,m)\})$
    \item Return $\sshare{x_{m-1}}$.
  \end{enumerate}
  \paragraph{MAMBA Example:} You can obtain the value of two raised to a secret shared number as follows:
  \begin{lstlisting}[language={python}]
a = sint(23)
l = 32
kappa = 20
# y stores the result of 2^23
y = AdvInteger.Pow2(a, l, kappa)
\end{lstlisting}

  \msubsubsection{$\mathsf{B2U}(\sshare{a},k, \kappa)$:}
  This converts the integer $a \in [0,\ldots,k)$ into unary form.
  It outputs $k$ bits, of which the last $a$ bits are zero to one,
  with the others set to zero.
  \begin{enumerate}
    \item $\sshare{2^a} \asn \mathsf{Pow2}(\sshare{a},k)$.
    \item $\sshare{r'}, \sshare{r}, \sshare{r}_B \asn \mathsf{PRandM}(k,k,\kappa)$.
    \item $c \asn \Open(\sshare{2^a}+\sshare{r}+2^k \cdot \sshare{r'})$.
    \item Let $c_{k-1},\ldots,c_0$ be the bits of $c$.
    \item For $i \in [0,\ldots,k=1]$ do
          \begin{enumerate}
            \item $\sshare{x_i} \asn c_i+\sshare{r_i}-2 \cdot c_i \cdot \sshare{r_i}$.
          \end{enumerate}
    \item $\sshare{y_{k-1}},\ldots,\sshare{y_0} \asn \mathsf{PreOp}(\mathsf{OR},\sshare{x_{k-1}},\ldots,\sshare{x_0},k)$.
    \item For $i \in [0,\ldots,k=1]$ do
          \begin{enumerate}
            \item $\sshare{a_i} \asn 1-\sshare{y_i}$.
          \end{enumerate}
    \item Return $\sshare{a_0},\ldots,\sshare{a_{k-1}}$.
  \end{enumerate}
  Note, in the function $\mathsf{Oblivious\_Trunc}$ below we also require the
  value $\sshare{2^a}$ to be returned so as to avoid recomputing it.

  \paragraph{MAMBA Example:} You can transform a number into its unary form as follows:
  \begin{lstlisting}[language={python}]
a = sint(3)
l=5
kappa=20
b,c = AdvInteger.B2U(a, l, kappa)
\end{lstlisting}


  \msubsubsection{$\mathsf{TruncPr}(\sshare{a},k,m, \kappa)$:}
  An approximate truncation algorithm which is faster than
  a fully accurate Trunc.
  Given $a \in \Zk$, $m \in [1,\ldots,k-1]$
  this outputs $\floor{a/2^m}+u$ where $u$ is a random
  (and unknown) bit.
  It gives the actual correct nearest integer with
  probability $1-\alpha$, where $\alpha$ is the distance
  between $a/2^m$ and that integer.
  \begin{enumerate}
    \item $\sshare{r'}, \sshare{r}, \sshare{r}_B \asn \mathsf{PRandM}(k,m,\kappa)$.
    \item $c \asn \Open(\sshare{a}+2^{k-1}+\sshare{r}+2^m \cdot \sshare{r'})$.
    \item $c'\asn c \pmod{2^m}$.
    \item $t \asn 1/2^m \pmod{p}$.
    \item $\sshare{d} \asn t \cdot (\sshare{a}-c'+\sshare{r})$.
    \item Return $\sshare{d}$.
  \end{enumerate}
  \paragraph{MAMBA Example:} A secret shared fractional register can be approximately truncated as follows:
  \begin{lstlisting}[language={python}]
a = sint(23)
k = 5
m = 3
kappa = 20
# where b is a register of type sint
b=AdvInteger.TruncPr(a, k, m, kappa)
\end{lstlisting}

  \msubsubsection{$\mathsf{Trunc}(\sshare{a},k,m, \kappa, \signed)$:}
  An exact version of $\mathsf{Trunc}$ above, it computes the value
$(a-a \mod{2^m})/2^m$, where the $\mod$ operator here produces a positive value.
  \begin{enumerate}
    \item $\sshare{a'} \asn \mathsf{Mod2m}(\sshare{a},k,m, \kappa, \signed)$.
    \item $t \asn 1/2^m \pmod{p}$.
    \item $\sshare{d} \asn t \cdot (\sshare{a}-\sshare{a'})$.
    \item Return $\sshare{d}$.
  \end{enumerate}
  Below we will give a version of $\mathsf{Trunc}$ in which $m$ is kept secret shared.

  \paragraph{MAMBA Example:} You truncate a number as follows, where a is a register of type \verb|sint|,
  \begin{lstlisting}[language={python}]
a = sint(23)
k = 5
m = 3
kappa = 20
a= AdvInteger.Trunc(a, k, m, kappa, True)
\end{lstlisting}
When the final value is set to $\true$ the value $a$ is assumed to lie in $\Zk$,
but when $\false$ the value $a$ is assumed to lie in $[0,\ldots,2^{k-1})$.

  There is also an exact version of $\mathsf{Trunc}$ called $\mathsf{TruncRoundNearest}$
  which rounds the value $a/2^m$ to the nearest integer.
  This is called as
  \begin{lstlisting}[language={python}]
a= AdvInteger.TruncRoundNearest(a, k, m, kappa)
\end{lstlisting}


  \msubsubsection{$\mathsf{Oblivious\_Trunc}(\sshare{a},k,\sshare{m}, \kappa)$:}
  This does the same operation as $\mathsf{Trunc}$ above, but $m$ is now secret shared, but guaranteed to be in the range $[0,\ldots,k)$.
  \begin{enumerate}
    \item $\sshare{x_0},\ldots,\sshare{x_{k-1}}, \sshare{2^m} \asn \mathsf{B2U}(\sshare{m},k)$.
    \item $\sshare{2^{-m}} \asn \mathsf{Inv}(\sshare{2^m})$.
    \item $\sshare{r''}, \sshare{r}, \sshare{r}_B \asn \mathsf{PRandM}(k,k,\kappa)$.
    \item $\sshare{r'} \asn \sum_{i=0}^{k-1} 2^i \cdot \sshare{x_i} \cdot \sshare{r_i}$
    \item $c \asn \Open(\sshare{a}+\sshare{r''} \cdot 2^{K} +\sshare{r})$.
    \item For $i \in [1,\ldots,k=1]$ do $c_i' \asn c \pmod{2^i}$.
    \item $\sshare{c''} \asn \sum_{i=1}^{k-1} c_i' \cdot (\sshare{x_{i-1}}-\sshare{x_i})$.
    \item $\sshare{d} \asn \mathsf{LT}(\sshare{c''},\sshare{r'},k)$.
    \item $\sshare{b} \asn (\sshare{a}-\sshare{c''}+\sshare{r'}) \cdot \sshare{2^{-m}} - \sshare{d}$.
    \item Return $\sshare{b}$.
  \end{enumerate}

  \paragraph{MAMBA Example:} You truncate a number as follows:
  \begin{lstlisting}[language={python}]
a = sint(23)
k = 5
m = sint(3)
kappa = 20
# where a is a register of type sint
b= AdvInteger.Oblivious_Trunc(a, k, m, kappa)
\end{lstlisting}

  \iffalse
    \note{Nigel}{Does not seem to be used anywhere}
    \msubsubsection{$\mathsf{Mod}(\sshare{a},k,x)$:}
    Given $a \in \Zk$ and public $x \in [1,\ldots,2^{k-1}-1]$
    this computes $\sshare{a \pmod{x}}$.
    \begin{enumerate}
      \item $m \asn \ceil{\log_2 x }$.
      \item $\sshare{r'}, \sshare{r}, \sshare{r}_B \asn \mathsf{PRandM}(k,m,\kappa)$.
      \item $c \asn \Open(\sshare{a}+2^{k-1}+\sshare{r}+x \cdot \sshare{r'})$.
      \item $c' \asn c \pmod{x}$.
      \item $\sshare{v} \asn 1 -\mathsf{BitLT}(\sshare{r}_B,x)$.
      \item $\sshare{u} \asn \mathsf{LTZ}(c' - \sshare{r} + x \cdot \sshare{v},m)$.
      \item $\sshare{a'} \asn c' - \sshare{r} + x \cdot ( \sshare{v}+\sshare{u})$.
      \item Return $\sshare{a'}$.
    \end{enumerate}
  \fi

  \msubsubsection{$\mathsf{LTZ}(\sshare{a},k, \kappa)$:}
  Given $a \in \Zk$ this tests whether $a<0$ or not,
  resulting in a shared bit.
  \begin{enumerate}
    \item $\sshare{s} \asn - \mathsf{Trunc}(\sshare{a},k,k-1)$.
  \end{enumerate}

  \paragraph{MAMBA Example:} You determine whether a number is less than zero as follows:
  \begin{lstlisting}[language={python}]
a = sint(1)
b = sint()
k=80
kappa=40
# b stores the result of x < 0
AdvInteger.LTZ(b, a, k, kappa)
\end{lstlisting}
  Like many commands in this section this can be abbreviated to
  \begin{lstlisting}[language={python}]
b=a<0
\end{lstlisting}
  In which case the default value of $\kappa=40$ is chosen
  (when using a $128$-bit prime modulus), this default value can be
  altered by using the command
  \begin{lstlisting}[language={python}]
program.security = 100
\end{lstlisting}
  The default value of $k=64$ is used in this setting, and
  this can be altered by executing
  \begin{lstlisting}[language={python}]
program.bit_length = 40
\end{lstlisting}
  The requirement is that $k+\kappa$ must be less than the bit length
  of the prime $p$.


  \msubsubsection{$\mathsf{EQZ}(\sshare{a},k, \kappa)$:}
  Given $a \in \Zk$ this tests whether $a=0$ or not,
  resulting in a shared bit.
  \begin{enumerate}
    \item $\sshare{r'}, \sshare{r}, \sshare{r}_B \asn \mathsf{PRandM}(k,k,\kappa)$.
    \item $c \asn \Open(\sshare{a}+2^{k-1}+2^k \cdot \sshare{r'}+\sshare{r})$.
    \item Let $c_{k-1},\ldots,c_0$ be the bits of $c$.
    \item For $i \in [0,\ldots,k-1]$ do
          \begin{enumerate}
            \item $\sshare{d_i} \asn c_i+\sshare{r_i}-2 \cdot c_i \cdot \sshare{r_i}$.
          \end{enumerate}
    \item $\sshare{z} \asn 1- \mathsf{KOp}(\mathsf{OR},\sshare{d_{k-1}},\ldots,\sshare{d_0},k)$.
    \item Return $\sshare{z}$.
  \end{enumerate}

  \paragraph{MAMBA Example:} You determine whether a number is equal to zero as follows:
  \begin{lstlisting}[language={python}]
a = sint(1)
b = sint()
k = 80
kappa = 40
# b stores the result of x == 0
AdvInteger.EQZ(b, a, k, kappa)
\end{lstlisting}


  \msubsubsection{Comparison Operators:}
  We can now define the basic comparison operators on shared
  representations from $\Zk$.
  \begin{center}
    \begin{tabular}{|lll|}
      \hline
      Operator & Protocol Name                        & Construction                            \\
      \hline
      $a>0$    & $\mathsf{GTZ(\sshare{a})}$           & $\mathsf{LTZ}(-\sshare{a})$             \\
      $a\le0$  & $\mathsf{LEZ(\sshare{a})}$           & $1-\mathsf{LTZ}(-\sshare{a})$           \\
      $a\ge0$  & $\mathsf{GEZ(\sshare{a})}$           & $1-\mathsf{LTZ}(\sshare{a})$            \\
      $a=b$    & $\mathsf{EQ(\sshare{a},\sshare{b})}$ & $\mathsf{EQZ}(\sshare{a}-\sshare{b})$   \\
      $a<b$    & $\mathsf{LT(\sshare{a,\sshare{b}})}$ & $\mathsf{LTZ}(\sshare{a}-\sshare{b})$   \\
      $a>b$    & $\mathsf{GT(\sshare{a,\sshare{b}})}$ & $\mathsf{LTZ}(\sshare{b}-\sshare{a})$   \\
      $a\le b$ & $\mathsf{LE(\sshare{a,\sshare{b}})}$ & $1-\mathsf{LTZ}(\sshare{b}-\sshare{a})$ \\
      $a\ge b$ & $\mathsf{GE(\sshare{a,\sshare{b}})}$ & $1-\mathsf{LTZ}(\sshare{a}-\sshare{b})$ \\
      \hline
    \end{tabular}
  \end{center}


  \iffalse
    ### NPS: This does not seem to be defined, or used
    \msubsubsection{$\mathsf{Mod2m}(a_{prime}, \sshare{a},k,\sshare{m}, \kappa)$:}
    This does the same operation as $\mathsf{Mod2m}$ above, but $m$ is now secret shared.
    \begin{enumerate}
      \item $\sshare{x_0},\ldots,\sshare{x_{k-1}}, \sshare{2^m} \asn \mathsf{B2U}(\sshare{m},k)$.
      \item $\sshare{2^{-m}} \asn \mathsf{Inv}(\sshare{2^m})$.
      \item $\sshare{r''}, \sshare{r}, \sshare{r}_B \asn \mathsf{PRandM}(k,k,\kappa)$.
      \item $\sshare{r'} \asn \sum_{i=0}^{k-1} 2^i \cdot \sshare{x_i} \cdot \sshare{r_i}$
      \item $c \asn \Open(\sshare{a}+\sshare{r''}+\sshare{r})$.
      \item For $i \in [1,\ldots,k=1]$ do $c_i' \asn c \pmod{2^i}$.
      \item $\sshare{c''} \asn \sum_{i=1}^{k-1} c_i' \cdot (\sshare{x_{i-1}}-\sshare{x_i})$.
      \item $\sshare{d} \asn \mathsf{LT}(\sshare{c''},\sshare{r'},k)$.
      \item $\sshare{b} \asn \sshare{c''}-\sshare{r'}+\sshare{2^m} \cdot \sshare{d}$.
      \item Return $\sshare{b}$.
    \end{enumerate}
  \fi

  \paragraph{MAMBA Example:} The \verb|mod| to a secret shared power of $2$ of a secret shared integer register can be obtain as follows:
  \begin{lstlisting}[language={python}]
a_prime = param_a_prime # a % 2^m
a sint(2137)
k = 16
m = sint(2)
kappa = 8
signed = True # True/False, describes a

# where sb is a register of type sint
r_dprime, r_prime, c, c_prime, u, t, c2k1 = \
    comparison.Mod2m(a_prime, a, k, m, kappa, signed)
\end{lstlisting}


  \msubsection{Arithmetic with Fixed Point Numbers}
  In this section we define basic arithmetic on fixed
  point numbers.
  We mainly follow the algorithms given in
  \begin{itemize}
    \item Secure Computation with Fixed-Point Numbers {\em FC 2010} \cite{CS10}.
  \end{itemize}
  We define $\Qk{f}$ as the set of rational numbers
$\{x \in \Q: x = \overline{x} \cdot 2^{-f}, \overline{x} \in \Zk\}$.
  We represent $x \in \Q$ as the integer $x \cdot 2^f = \overline{x} \in \Zk$,
  which is then represented in $\F_p$ via the mapping used above.
  Thus $x \in \Q$ is in the range $[-2^e,2^e-2^{-f}]$
  where $e=k-f$.
  As we are working with fixed point numbers we assume that the
  parameters $f$ and $k$ are public.
  For our following algorithms to work (in particular fixed point
  multiplication and division) we require that
$f<k$ and $2 \cdot k + \kappa < \log_2 p$.
  By abuse of notation we write $\sshare{a}$ to mean $\sshare{\overline{a}}$,
  i.e. the secret sharing of the fixed point number $a$,
  is actually the secret sharing of the integer representative
$\overline{a}$.

  \iffalse
    %NPS Does not seem to be used anywhere
    \msubsubsection{$\mathsf{Scale}(\sshare{a},k,f_1,f_2)$:}
    Sometimes we want to scale the input fixed point number $a$
    from $\Qk{f_1}$ to $\Qk{f_2}$.
    \begin{enumerate}
      \item $m \asn f_2-f_1$.
      \item If $m\ge 0$ then $\sshare{a'} \asn 2^m \cdot \sshare{a}$.
      \item Else $\sshare{a'} \asn \mathsf{TruncPr}(\sshare{a},k,-m)$.
      \item Return $\sshare{a'}$.
    \end{enumerate}
    This is not directly callable from MAMBA it is here purely
    for documentation reasons.
  \fi

  \msubsubsection{$\mathsf{FxEQZ}, \mathsf{FxLTZ}, \mathsf{FxEQ}, \mathsf{FxLT}$, etc:}
  All of the comparison operators for integers given above carry
  over to fixed point numbers (if the inputs have the same $f$-values).

  \msubsubsection{$\mathsf{FxAbs}(\sshare{a},k,f)$}
  \begin{enumerate}
    \item $\sshare{s} \asn \mathsf{LTZ}(\sshare{a})$.
    \item $\sshare{a} \asn (1-2\cdot \sshare{s}) \cdot \sshare{a}$.
    \item Return $\sshare{a}$.
  \end{enumerate}

  \paragraph{MAMBA Example:} To obtain the absolute value of a number as follows:
  \begin{lstlisting}[language={python}]
b = -1.5
sb = sfix(b)

# k and f are extracted from b
# returns unsigned b and receives signed b
ub = FixedPt.abs_fx(sb)
\end{lstlisting}

  \msubsubsection{$\mathsf{FxFloor}(\sshare{a},k,f)$}
  \begin{enumerate}
    \item $\sshare{s} \asn \mathsf{Trunc}(\sshare{a}, k-f, f, \kappa)$.
    \item Return $\sshare{s}$.
  \end{enumerate}

  \paragraph{MAMBA Example:} To obtain the floor of an \verb+sfix+ value
  as an \verb+sfix+value
  \begin{lstlisting}[language={python}]
b = -1.5
sb = sfix(b)

ub = FixedPt.floor_fx(sb)
\end{lstlisting}

  \msubsubsection{$\mathsf{FxNeg}(\sshare{a},k,f)$}
  \begin{enumerate}
    \item Return $-\sshare{a}$.
  \end{enumerate}
  \paragraph{MAMBA Example:} To obtain the original value times $-1$ as follows:
  \begin{lstlisting}[language={python}]
b = -1.5
sb = sfix(b)
# k and f are extracted from b
# returns the value of signed b times -1.
nb = -sb
\end{lstlisting}
  \msubsubsection{$\mathsf{FxAdd}(\sshare{a},\sshare{b},k,f)$:}
  Given $a, b \in \Qk{f}$ this is just the integer addition algorithm for elements
  in $\Zk$ given above
  \begin{enumerate}
    \item $\sshare{c} \asn \sshare{a} + \sshare{b}$.
    \item Return $\sshare{c}$.
  \end{enumerate}
  Obviously from $\mathsf{FxNeg}$ and $\mathsf{FxAdd}$ we can define $\mathsf{FxSub}$.
  Note that, if the sum overflows then the resulting value will be
  invalid.
  \paragraph{MAMBA Example:} To obtain the addition of two fixed point secret shared values you can do as follows:
  \begin{lstlisting}[language={python}]
a = sfix(3.5)
b = sfix(1.5)
# k and f are extracted from b or a
#returns secret shared 5
a_plus_b = a+b
\end{lstlisting}
  \msubsubsection{$\mathsf{FxMult}(\sshare{a},\sshare{b},k,f)$:}
  Given $a, b \in \Qk{f}$ this requires integer multiplication followed
  by a suitable truncation.
  \begin{enumerate}
    \item $\sshare{d} \asn \sshare{a} \cdot \sshare{b}$.
    \item $\sshare{c} \asn \mathsf{TrunkPR}(\sshare{d},2 \cdot k,f)$.
    \item Return $\sshare{c}$.
  \end{enumerate}
  \paragraph{MAMBA Example:} To obtain the multiplication of two fixed point secret shared values you can do as follows:
  \begin{lstlisting}[language={python}]
a = sfix(3.5)
b = sfix(1.5)
# k and f are extracted from b or a
#returns secret shared 5.25
a_mult_b = a*b
\end{lstlisting}

  \msubsubsection{$\mathsf{FxDiv}(\sshare{a},b,k,f)$:}
  We first give division for when $a, b \in \Qk{f}$ and $b$ is in the clear.
  \begin{enumerate}
    \item Compute $x \in \Qk{f}$ such that $x \approx 1/b$.
    \item $\sshare{y} \asn \mathsf{TruncPr}(\overline{x} \cdot \sshare{a},k,f)$.
    \item Return $\sshare{y}$.
  \end{enumerate}

  \paragraph{MAMBA Example:} To divide two fixed point values (where only one is secret shared) you can do as follows:
  \begin{lstlisting}[language={python}]
a = sfix(3.5)
b = 1.5
# k and f are extracted from b
#returns secret shared 2.333333
a_div_b = a/b
\end{lstlisting}
  \msubsubsection{$\mathsf{FxDiv}(\sshare{a},\sshare{b},k,f)$:}
  This operation is more complex and we use method of Goldschmidt, which
  is recommended by Catrina et. al.
  The following routine makes use of the two subroutines which follow
  \begin{enumerate}
    \item $\theta \asn \ceil{\log_2 (k/3.5)}$.
    \item $\overline{\alpha} \asn 2^{2 \cdot f}$.
          Note that $\overline{\alpha}$ is the integer representative of $1.0$ in $\Qk{2\cdot f}$.
    \item $\sshare{w} \asn \mathsf{AppRcr}(\sshare{b},k,f)$.
    \item $\sshare{x} \asn \overline{\alpha} - \sshare{b}\cdot \sshare{w}$.
    \item $\sshare{y} \asn \sshare{a}\cdot \sshare{w}$.
    \item $\sshare{y} \asn \mathsf{TruncPr}(\sshare{y},2\cdot k,f)$.
    \item For $i \in [1,\ldots,\theta-1]$ do
          \begin{enumerate}
            \item $\sshare{y} \asn \sshare{y} \cdot (\overline{\alpha}+\sshare{x})$.
            \item $\sshare{x} \asn \sshare{x}^2$.
            \item $\sshare{y} \asn \mathsf{TruncPr}(\sshare{y},2\cdot k,2 \cdot f)$.
            \item $\sshare{x} \asn \mathsf{TruncPr}(\sshare{x},2\cdot k,2 \cdot f)$.
          \end{enumerate}
    \item $\sshare{y} \asn \sshare{y} \cdot (\overline{\alpha}+\sshare{x})$.
    \item $\sshare{y} \asn \mathsf{TruncPr}(\sshare{y},2\cdot k,2 \cdot f)$.
    \item Return $\sshare{y}$.
  \end{enumerate}
  \paragraph{MAMBA Example:} To obtain the division of two fixed point values that are secret shared, you can do as follows:
  \begin{lstlisting}[language={python}]
a = sfix(3.5)
b = 1.5
# k and f are extracted from b
#returns secret shared 2.333333
a_div_b = a/b
\end{lstlisting}

  \msubsubsection{$\mathsf{AppRcr}(\sshare{b},k,f)$:}
  \begin{enumerate}
    \item $\overline{\alpha} \asn 2.9142 \cdot 2^k$.
          Note that $\overline{\alpha}$ is the integer representative of $2.9142$ in $\Qk{f}$.
    \item $(\sshare{c},\sshare{v}) \asn \mathsf{Norm}(\sshare{b},k,f)$.
    \item $\sshare{d} \asn \overline{\alpha} - 2 \cdot \sshare{c}$.
    \item $\sshare{w} \asn \sshare{d} \cdot \sshare{v}$.
    \item $\sshare{w} \asn \mathsf{TruncPr}(\sshare{w},2 \cdot k,2 \cdot (k-f))$.
    \item Return $\sshare{w}$.
  \end{enumerate}
  This is not mean to be called from MAMBA; but
  if you insist it is
  \begin{lstlisting}[language={python}]
a = sfix(3.5)
b = FixedPt.AppRcr(a, k, f, kappa)
\end{lstlisting}

  \msubsubsection{$\mathsf{Norm}(\sshare{b},k,f):$}
  This returns the value $c$ such that $2^{k-1} \le c <2^k$
  and $v'$ such that $b \cdot v' = c$,
  and if $2^{m-1} \le |b| <2^m$ then $v'= \pm 2^{k-m}$.
  \begin{enumerate}
    \item $\sshare{s} \asn 1-2 \cdot \mathsf{LTZ}(\sshare{b},k)$.
    \item $\sshare{x} \asn \sshare{s} \cdot \sshare{b}$.
    \item $\sshare{x_{k-1}},\ldots,\sshare{x_0} \asn \mathsf{BitDec}(\sshare{x},k,k)$.
    \item $\sshare{y_{k-1}},\ldots,\sshare{y_0} \asn \mathsf{PreOp}(\mathsf{OR},\sshare{x_{k-1}},\ldots,\sshare{x_0},k)$.
    \item For $i \in [0,\ldots,k-2]$ do
          \begin{enumerate}
            \item $\sshare{z_i} \asn \sshare{y_i}-\sshare{y_{i+1}}$.
          \end{enumerate}
    \item $\sshare{z_{k-1}} \asn \sshare{y_{k-1}}$.
    \item $\sshare{v} \asn \sum_{i=0}^{k-1} 2^{k-i-1} \cdot \sshare{z_i}$.
    \item $\sshare{c} \asn \sshare{x} \cdot \sshare{v}$.
    \item $\sshare{v'} \asn \sshare{s} \cdot \sshare{v}$.
    \item Return $(\sshare{c}, \sshare{v'})$.
  \end{enumerate}
  \paragraph{MAMBA Example:} To obtain the norm of a secret shared fix point register you can do as follows:
  \begin{lstlisting}[language={python}]
kappa = 40
b = sfix(1.5)
#returns the norm
c, v = FixedPt.Norm(b, b.k, b.f, kappa, True)
\end{lstlisting}

  \msubsubsection{$\mathsf{NormSQ}(\sshare{b},k):$}
  As above, but now we assume $b \ge 0$, and we also
  output shares of $m$ and $w$ such that $w=2^{m/2}$
  if $m$ is even and $2^{(m-1)/2}$ if $m$ is odd.
  This algorithm is used in the Fixed Point sqrt routine later.
  Furthermore $v = 2^{k-m}$. Note that we have introduced some adaptations from the original paper:
  \begin{enumerate}
    \item $z \asn \mathsf{MSB}(\sshare{b},k,f)$.
    \item $\sshare{v} \asn \sum_{i=0}^{k-1} 2^{k-i-1} \cdot \sshare{z_i}$.
    \item $\sshare{c} \asn \sshare{b} \cdot \sshare{v}$.
    \item $\sshare{m} \asn \sum_{i=0}^{k-1} (i+1) \cdot \sshare{z_i}$.
    \item For $i \in [1,\ldots,k/2)$ do
          \begin{enumerate}
            \item $\sshare{w_i}=\sshare{z_{2\cdot i-1}}+\sshare{z_{2\cdot i}}$.
          \end{enumerate}
    \item $\sshare{w} \asn \sum_{i=1}^{k/2-1} 2^i \cdot \sshare{w_i}$.
    \item Return $\sshare{c}, \sshare{v}, \sshare{m}, \sshare{w}$.
  \end{enumerate}

  \msubsubsection{$\mathsf{SimplifiedNormSQ}(\sshare{b},k):$}
  Same as above, but in this case we only return $w$, together with a $\{0,1\}$ bit signaling whether $m$ is odd.
  This algorithm is used in the Fixed Point sqrt routine later.

  \begin{enumerate}
    \item $z \asn \mathsf{MSB}(\sshare{b},k,f)$.

    \item $\sshare{m} \asn \sum_{i=0}^{k-1} (i+1) \cdot \sshare{z_i}$.
    \item For $i \in [0,\ldots,k-1]$ do
          \begin{enumerate}
            \item If $(i \quad \% \quad 2 == 0)$:
                  \begin{enumerate}
                    \item $\sshare{m_{odd}} \asn m_{odd} + \sshare{z_{i}}$
                  \end{enumerate}
          \end{enumerate}
    \item For $i \in [1,\ldots,k/2)$ do
          \begin{enumerate}
            \item $\sshare{w_i}=\sshare{z_{2\cdot i-1}}+\sshare{z_{2\cdot i}}$.
          \end{enumerate}
    \item $\sshare{w} \asn \sum_{i=1}^{k/2-1} 2^i \cdot \sshare{w_i}$.

    \item Return $\sshare{m_{odd}}, \sshare{w}$.
  \end{enumerate}

  \msubsubsection{$\mathsf{MSB}(\sshare{b},k):$}
  Returns index array $\sshare{z}$ of size $k$, such that it holds a $1$ in the position of the most significative bit of $\sshare{b}$ and $\sshare{0}$ otherwise. This function is used internally in \verb|NormSQ| and \verb|SimplifiedNormSQ|.
  \begin{enumerate}

    \item $\sshare{s} \asn 1-2 \cdot \mathsf{LTZ}(\sshare{b},k)$.
    \item $\sshare{x} \asn \sshare{s} \cdot \sshare{b}$.
    \item $\sshare{x_{k-1}},\ldots,\sshare{x_0} \asn \mathsf{BitDec}(\sshare{x},k,k)$.
    \item $\sshare{y_{k-1}},\ldots,\sshare{y_0} \asn \mathsf{PreOp}(\mathsf{OR},\sshare{x_{k-1}},\ldots,\sshare{x_0},k)$.
    \item For $i \in [0,\ldots,k-1]$ do
          \begin{enumerate}
            \item $\sshare{z_i} \asn \sshare{y_i}-\sshare{y_{i+1}}$.
          \end{enumerate}
    \item $\sshare{z_{k-1}} \asn \sshare{y_{k-1}}$.
    \item Return $\sshare{z}$.
  \end{enumerate}


  \msubsection{Arithmetic with Floating Point Numbers}

  For floating point numbers we utilize the methods described in
  \begin{itemize}
    \item Secure Computation on Floating Point Numbers {\em NDSS 2013} \cite{ABZS13}.
  \end{itemize}
  However we make explicit use of an error flag which we carry throughout
  a computation, as detailed in previous sections.
  The processing of overflow and underflow detection is expensive,
  and thus we enable the user to turn this off via
  means of a compile time flag \verb+fdflag+ by passing the option
  \verb+-f+ or \verb+--fdflag+ when using \verb|compile.py|.
  This can also be turned on/off within a program by assigning
  to the variable \verb+program.fdflag+.
  The error flag is still needed however to catch other forms of errors
  in computations (such as division by zero, taking square roots of
  negative numbers etc).
  Thus setting \verb+fdflag+ equal to $\false$
  does not necessarily result in $\err$ always equaling zero.

  Floating point numbers are defined by two global, public integer parameters
$(\ell,k)$ which define the size of the mantissa and the exponent respectively.
  Each floating point number is represented as a five tuple $(v,p,z,s,\err)$, where
  \begin{itemize}
    \item $v \in [2^{\ell-1},2^\ell)$ is an $\ell+1$-bit significand with it's most
          significant bit always set to one.
    \item $p \in \Zk$ is the signed exponent.
    \item $z$ is a bit to define whether the number is zero or not.
    \item $s$ is a sign bit (equal to zero if non-negative).
    \item $\err$ is the error flag (equal to zero if no error has occurred, it holds a non-zero value otherwise).
  \end{itemize}
  Thus assuming $\err=0$ this tuple represents the value
  \[ u=(1- 2 \cdot s) \cdot (1-z) \cdot v \cdot 2^p. \]
  We adopt the conventions that when $u=0$ we also have $z=1, v=0$ and $p=0$,
  and when $\err \ne 0$ then the values of $v,p,z$ and $s$ are meaningless.

  The standard arithmetic operations of addition, multiplication and
  comparison (plus others) are then implemented in MAMBA for this datatype using
  operator overloading. The precise algorithms which are executed
  are detailed below.

  \msubsubsection{$\mathsf{FlowDetect}(\sshare{p})$:}
  \begin{enumerate}
    \item If $\mathsf{fdflag}$ then
          \begin{enumerate}
            \item $\sshare{s} \asn -2 \cdot (\sshare {p} < 0) + 1$.
                  %\item $\sshare{of} \asn \mathsf{GT}(\sshare{p},2^{k-1}-1,k+1)$.
                  %\item $\sshare{uf} \asn \mathsf{LT}(\sshare{p},-2^{k-1}-1,k+1)$.
            \item $\sshare{\err} \asn \mathsf{GT}(\sshare{p} \cdot \sshare{s},2^{k-1}-1,k+1)$.
                  %\sshare{of} + \sshare{uf}.
          \end{enumerate}
    \item Return $\sshare{\err}$.
  \end{enumerate}

  \msubsubsection{$\mathsf{FLNeg}((\sshare{v},\sshare{p},\sshare{z},\sshare{s},\sshare{\err}))$:}
\begin{enumerate}
  \item $\sshare{s} \asn 1-\sshare{s}$.
  \item Return $(\sshare{v},\sshare{p},\sshare{z},\sshare{s},\sshare{\err})$
\end{enumerate}

\msubsubsection{$\mathsf{FLAbs}((\sshare{v},\sshare{p},\sshare{z},\sshare{s},\sshare{\err}))$:}
  \begin{enumerate}
    \item $\sshare{s} \asn 0$.
    \item Return $(\sshare{v},\sshare{p},\sshare{z},\sshare{s},\sshare{\err})$
  \end{enumerate}
  Obviously from $\mathsf{FLNeg}$ and $\mathsf{FLAdd}$ we can define $\mathsf{FLSub}$.

  \msubsubsection{$\mathsf{FLMult}(
(\sshare{v_1},\sshare{p_1},\sshare{z_1},\sshare{s_1},\sshare{\err_1}),
(\sshare{v_2},\sshare{p_2},\sshare{z_2},\sshare{s_2},\sshare{\err_2}))$:}
For floating point operations multiplication is much easier than addition, so we deal with
this first.
\begin{enumerate}
  \item $\sshare{v} \asn \sshare{v_1} \cdot \sshare{v_2}$.
  \item $\sshare{v} \asn \mathsf{Trunc}(\sshare{v},2 \cdot \ell, \ell-1)$.
  \item $\sshare{b} \asn \mathsf{LT}(\sshare{v},2^\ell,\ell+1)$.
  \item $\sshare{v'} \asn \sshare{v}+\sshare{b} \cdot \sshare{v}$.
  \item $\sshare{v} \asn \mathsf{Trunc}(\sshare{v'}, \ell+1, 1)$.
  \item $\sshare{z} \asn \mathsf{OR}(\sshare{z_1},\sshare{z_2})$.
  \item $\sshare{s} \asn \mathsf{XOR}(\sshare{s_1},\sshare{s_2})$.
  \item $\sshare{p} \asn (\sshare{p_1}+\sshare{p_2}+\ell-\sshare{b}) \cdot (1-\sshare{z})$.
  \item $\sshare{\err} \asn \sshare{\err_1} + \sshare{\err_2} + \mathsf{FlowDetect}(\sshare{p})$.
  \item Return $(\sshare{v},\sshare{p},\sshare{z},\sshare{s},\sshare{\err})$.
\end{enumerate}


\msubsubsection{$\mathsf{FLAdd}(
  (\sshare{v_1},\sshare{p_1},\sshare{z_1},\sshare{s_1},\sshare{\err_1}),
  (\sshare{v_2},\sshare{p_2},\sshare{z_2},\sshare{s_2},\sshare{\err_2}))$:}
  \begin{enumerate}
    \item $\sshare{a} \asn \mathsf{LT}(\sshare{p_1},\sshare{p_2},k)$.
    \item $\sshare{b} \asn \mathsf{EQ}(\sshare{p_1},\sshare{p_2},k)$.
    \item $\sshare{c} \asn \mathsf{LT}(\sshare{v_1},\sshare{v_2},k)$.
    \item $\sshare{p_{\max}} \asn \sshare{a} \cdot \sshare{p_2} + (1-\sshare{a}) \cdot \sshare{p_1}$.
    \item $\sshare{p_{\min}} \asn (1-\sshare{a}) \cdot  \sshare{p_2} + \sshare{a} \cdot \sshare{p_1}$.
    \item $\sshare{a \cdot b} \asn \sshare{a} \cdot \sshare{b}$.
    \item $\sshare{b \cdot c} \asn \sshare{b} \cdot \sshare{c}$.
    \item $\sshare{v_{\max}} \asn
            (\sshare{a \cdot b}-\sshare{a}-\sshare{b \cdot c}) \cdot (\sshare{v_1}-\sshare{v_2})
            + \sshare{v_1}$.
    \item $\sshare{v_{\min}} \asn
            (\sshare{a \cdot b}-\sshare{a}-\sshare{b \cdot c}) \cdot (\sshare{v_2}-\sshare{v_1})
            + \sshare{v_2}$.
    \item $\sshare{s_3} \asn \mathsf{XOR}(\sshare{s_1},\sshare{s_2})$.
    \item $\sshare{d} \asn \mathsf{LT}(\ell, \sshare{p_{\max}}-\sshare{p_{\min}}, k)$.
    \item $\sshare{2^\Delta} \asn \mathsf{Pow2}((1-\sshare{d}) \cdot (\sshare{p_{\max}}-\sshare{p_{\min}),\ell+1)$.
    \item $\sshare{v_3} \asn 2 \cdot (\sshare{v_{\max}}-\sshare{s_3})+1$.
    \item $\sshare{v_4} \asn \sshare{v_{\max}} \cdot \sshare{2^\Delta}+(1-2 \cdot \sshare{s_3}) \cdot \sshare{v_{\min}}$.
    \item $\sshare{v} \asn (\sshare{d} \cdot \sshare{v_3}+(1-\sshare{d}) \cdot \sshare{v_4})
            \cdot 2^\ell \cdot \mathsf{Inv}(\sshare{2^\Delta}})$.
    \item $\sshare{v} \asn \mathsf{Trunc}(\sshare{v},2 \cdot \ell+1,\ell-1)$.
    \item $\sshare{u_{\ell+1}},\ldots,\sshare{u_0} \asn \mathsf{BitDec}(\sshare{v},\ell+2,\ell+2)$.
    \item $\sshare{h_0},\ldots,\sshare{h_{\ell+1}} \asn \mathsf{PreOp}(\mathsf{OR},\sshare{u_{\ell+1}},\ldots,\sshare{u_0},k)$.
    \item $\sshare{p_0} \asn \ell+2-\sum_{i=0}^{\ell+1} \sshare{h_i}$.
    \item $\sshare{2^{p_0}} \asn 1+ \sum_{i=0}^{\ell+1} 2^i \cdot (1-\sshare{h_i})$.
    \item $\sshare{v} \asn \mathsf{Trunc}(\sshare{2^{p_0}} \cdot \sshare{v},\ell+2,2)$.
    \item $\sshare{p} \asn \sshare{p_{\max}}-\sshare{p_0}+1-\sshare{d}$.
    \item $\sshare{z_1 \cdot z_2} \asn \sshare{z_1} \cdot \sshare{z_2}$.
    \item $\sshare{v} \asn (1-\sshare{z_1}-\sshare{z_2} +\sshare{z_1 \cdot z_2}) \cdot \sshare{v}
            + \sshare{z_1} \cdot \sshare{v_2} + \sshare{z_2} \cdot \sshare{v_1}$.
    \item $\sshare{z} \asn \mathsf{EQZ}(\sshare{v},\ell)$.
    \item $\sshare{p} \asn (1-\sshare{z_1}-\sshare{z_2}+\sshare{z_1 \cdot z_2}) \cdot \sshare{p}
            + \sshare{z_1} \cdot \sshare{p_2}
            + \sshare{z_2} \cdot \sshare{p_1}) \cdot (1-\sshare{z})$.
    \item $\sshare{s} \asn (\sshare{a}-\sshare{a \cdot b}) \cdot \sshare{s_2}
            + (1-\sshare{a}-\sshare{b}+\sshare{a \cdot b}) \cdot \sshare{s_1})
            + \sshare{b \cdot c} \cdot \sshare{s_2}
            + (\sshare{b}-\sshare{b \cdot c}) \cdot \sshare{s_1})$.
    \item $\sshare{s} \asn (1-\sshare{z_1}-\sshare{z_2}+\sshare{z_1 \cdot z_2}) \cdot \sshare{s}
            + (\sshare{z_2}-\sshare{z_1 \cdot z_2}) \cdot \sshare{s_1}
            + (\sshare{z_1}-\sshare{z_1 \cdot z_2}) \cdot \sshare{s_2}$.
    \item $\sshare{\err} \asn \sshare{\err_1} + \sshare{\err_2} + \mathsf{FlowDetect}(\sshare{p})$.
    \item Return $(\sshare{v},\sshare{p},\sshare{z},\sshare{s},\sshare{\err})$.
  \end{enumerate}

  \msubsubsection{$\mathsf{SDiv}(\sshare{a},\sshare{b},\ell)$:}
  \begin{enumerate}
    \item $\theta \asn \ceil{\log_{2} \ell}$.
    \item $\sshare{x} \asn \sshare{b}$.
    \item $\sshare{y} \asn \sshare{a}$.
    \item For $i \in [1,\ldots,\theta-1]$ do
          \begin{enumerate}
            \item $\sshare{y} \asn \sshare{y} \cdot (2^{\ell+1}-\sshare{x})$.
            \item $\sshare{y} \asn \mathsf{TruncPr}(\sshare{y},2\cdot \ell+1,\ell)$.
            \item $\sshare{x} \asn \sshare{x} \cdot (2^{\ell+1}-\sshare{x})$.
            \item $\sshare{x} \asn \mathsf{TruncPr}(\sshare{x},2\cdot \ell+1,\ell)$.
          \end{enumerate}
    \item $\sshare{y} \asn \sshare{y} \cdot (2^{\ell+1}-\sshare{x})$.
    \item $\sshare{y} \asn \mathsf{TruncPr}(\sshare{y},2\cdot \ell+1,\ell)$.
    \item Return $\sshare{y}$.
  \end{enumerate}

  \msubsubsection{$\mathsf{FLDiv}(
(\sshare{v_1},\sshare{p_1},\sshare{z_1},\sshare{s_1},\sshare{\err_1}),
(\sshare{v_2},\sshare{p_2},\sshare{z_2},\sshare{s_2},\sshare{\err_2}))$:}
\begin{enumerate}
  \item $\sshare{v} \asn \mathsf{SDiv}(\sshare{v_1},\sshare{v_2}+\sshare{z_2},\ell)$.
  \item $\sshare{b} \asn \mathsf{LT}(\sshare{v},2^\ell,\ell+1)$.
  \item $\sshare{v'} \asn \sshare{v}+\sshare{b} \cdot \sshare{v}$.
  \item $\sshare{v} \asn \mathsf{Trunc}(\sshare{v'}, \ell+1, 1)$.
  \item $\sshare{z} \asn \sshare{z_1}$.
  \item $\sshare{s} \mathsf{XOR}(\sshare{s_1},\sshare{s_2})$.
  \item $\sshare{p} \asn (\sshare{p_1}-\sshare{p_2}-\ell+1-\sshare{b}) \cdot (1-\sshare{z})$.
  \item $\sshare{\err} \asn \sshare{\err_1} + \sshare{\err_2}$.
  \item $\sshare{\err} \asn \sshare{\err} + \sshare{z_2}$.
  \item $\sshare{\err} \asn \sshare{\err} + \mathsf{FlowDetect}(\sshare{p})$.
  \item Return $(\sshare{v},\sshare{p},\sshare{z},\sshare{s},\sshare{\err})$.
\end{enumerate}


\msubsubsection{$\mathsf{FLLTZ}((\sshare{v},\sshare{p},\sshare{z},\sshare{s},\sshare{\err}))$:}
  \begin{enumerate}
    \item Return $\sshare{s} \cdot (1-\sshare{z}) \cdot \mathsf{EQZ}(\sshare{\err}, k)$.
  \end{enumerate}

  \msubsubsection{$\mathsf{FLEQZ}((\sshare{v},\sshare{p},\sshare{z},\sshare{s},\sshare{\err}))$:}
\begin{enumerate}
  \item Return $\sshare{z} \cdot \mathsf{EQZ}(\sshare{\err}, k)$.
\end{enumerate}

\msubsubsection{$\mathsf{FLGTZ}((\sshare{v},\sshare{p},\sshare{z},\sshare{s},\sshare{\err}))$:}
  \begin{enumerate}
    \item Return $(1-\sshare{s}) \cdot (1-\sshare{z}) \mathsf{EQZ}(\sshare{\err}, k)$.
  \end{enumerate}

  \msubsubsection{$\mathsf{FLLEZ}((\sshare{v},\sshare{p},\sshare{z},\sshare{s},\sshare{\err}))$:}
\begin{enumerate}
  \item Return $\sshare{s} \cdot (1 - \mathsf{EQZ}(\sshare{\err}, k))$.
\end{enumerate}

\msubsubsection{$\mathsf{FLGEZ}((\sshare{v},\sshare{p},\sshare{z},\sshare{s},\sshare{\err}))$:}
  \begin{enumerate}
    \item Return $(1-\sshare{s}) \cdot \mathsf{EQZ}(\sshare{\err}, k)$.
  \end{enumerate}

  \msubsubsection{$\mathsf{FLEQ}(
(\sshare{v_1},\sshare{p_1},\sshare{z_1},\sshare{s_1},\sshare{\err_1}),
(\sshare{v_2},\sshare{p_2},\sshare{z_2},\sshare{s_2},\sshare{\err_2}))$:}
\begin{enumerate}
  \item $\sshare{b_1} \asn \mathsf{EQ}(\sshare{v_1},\sshare{v_2},\ell)$.
  \item $\sshare{b_2} \asn \mathsf{EQ}(\sshare{p_1},\sshare{p_2},k)$.
  \item $\sshare{b_3} \asn \sshare{z_1} \cdot \sshare{z_2}$.
  \item $\sshare{b_4} \asn \sshare{s_1} \cdot \sshare{s_2}$.
  \item $\sshare{t} \asn \sshare{err_1} + \sshare{err_2}$.
  \item $\sshare{t} \asn (\mathsf{EQZ}(\sshare{t}, k)$.
  \item Return $(\sshare{b_1} \cdot \sshare{b_2} \cdot \sshare{b_3} \cdot (1 - \sshare{b_4}) + \sshare{b_4}) \cdot \sshare{t}$.
\end{enumerate}

\msubsubsection{$\mathsf{FLLT}(
  (\sshare{v_1},\sshare{p_1},\sshare{z_1},\sshare{s_1},\sshare{\err_1}),
  (\sshare{v_2},\sshare{p_2},\sshare{z_2},\sshare{s_2},\sshare{\err_2}))$:}
  \begin{enumerate}
    \item $\sshare{a} \asn \mathsf{LT}(\sshare{p_1},\sshare{p_2},k)$.
    \item $\sshare{c} \asn \mathsf{EQ}(\sshare{p_1},\sshare{p_2},k)$.
    \item $\sshare{d} \asn \mathsf{LT}((1-2\cdot \sshare{s_1}) \cdot \sshare{v_1},
            (1-2\cdot \sshare{s_2}) \cdot \sshare{v_2},\ell+1)$.
    \item $\sshare{a \cdot c} \asn \sshare{a}\cdot \sshare{c}$.
    \item $\sshare{c \cdot d} \asn \sshare{c}\cdot \sshare{d}$.
    \item $\sshare{b^+} \asn \sshare{c \cdot d}+(\sshare{a}-\sshare{a \cdot c})$.
    \item $\sshare{b^-} \asn \sshare{c \cdot d}+(1-\sshare{c}-\sshare{a}+\sshare{a \cdot c})$.
    \item $\sshare{z_1 \cdot z_2} \asn \sshare{z_1}\cdot \sshare{z_2}$.
    \item $\sshare{s_1 \cdot s_2} \asn \sshare{s_1}\cdot \sshare{s_2}$.
    \item $\sshare{b} \asn \sshare{z_1 \cdot z_2} \cdot (\sshare{s_2}-1-\sshare{s_1 \cdot s_2})
            + \sshare{s_1 \cdot s_2} \cdot (\sshare{z_1}+\sshare{z_2}-1)
            + \sshare{z_1}\cdot (1-\sshare{s1}-\sshare{s_2})
            + \sshare{s_1}$.
    \item $\sshare{t} \asn \sshare{err_1} + \sshare{err_2}$.
    \item $\sshare{t} \asn \mathsf{EQZ}(\sshare{t}, k)$.
    \item $\sshare{b} \asn \sshare{b}+
            (1-\sshare{z_1}-\sshare{z_2}+\sshare{z_1 \cdot z_2} \cdot
            ((1-\sshare{s_1}-\sshare{s_2}+\sshare{s_1 \cdot s_2}) \cdot \sshare{b^+}
            +\sshare{s_1 \cdot s_2} \cdot \sshare{b^-}) \cdot \sshare{t}$.
    \item Return $\sshare{b}$.
  \end{enumerate}


  \msubsubsection{$\mathsf{FLGT}(
(\sshare{v_1},\sshare{p_1},\sshare{z_1},\sshare{s_1},\sshare{\err_1}),
(\sshare{v_2},\sshare{p_2},\sshare{z_2},\sshare{s_2},\sshare{\err_2}))$:}
\begin{enumerate}
  \item $\sshare{v_r}, \sshare{p_r}, \sshare{z_r}, \sshare{s_r}, \sshare{err_r} \asn
          \mathsf{FLAdd}(
          (\sshare{v_2},\sshare{p_2},\sshare{z_2},1 -\sshare{s_2},\sshare{\err_2}),
          (\sshare{v_1},\sshare{p_1},\sshare{z_1},\sshare{s_1},\sshare{\err_1}))$.
  \item Return $\mathsf{FLLTZ}(\sshare{v_r}, \sshare{p_r}, \sshare{z_r}, \sshare{s_r}, \sshare{err_r})$.
\end{enumerate}

\msubsubsection{$\mathsf{FLLET}(
  (\sshare{v_1},\sshare{p_1},\sshare{z_1},\sshare{s_1},\sshare{\err_1}),
  (\sshare{v_2},\sshare{p_2},\sshare{z_2},\sshare{s_2},\sshare{\err_2}))$:}
  \begin{enumerate}
    \item $\sshare{b} \asn \mathsf{FLGT}(
            (\sshare{v_1},\sshare{p_1},\sshare{z_1},\sshare{s_1},\sshare{\err_1}),
            (\sshare{v_2},\sshare{p_2},\sshare{z_2},\sshare{s_2},\sshare{\err_2}))$.
          %\item $\sshare{b_2} \asn \mathsf{FLEQ}(
          %        (\sshare{v_1},\sshare{p_1},\sshare{z_1},\sshare{s_1},\sshare{\err_1}),
          %        (\sshare{v_2},\sshare{p_2},\sshare{z_2},\sshare{s_2},\sshare{\err_2}))$.
    \item Return $1 - \sshare{b}$%$\mathsf{OR}(\sshare{b_1},\sshare{b_2})$.
  \end{enumerate}

  \msubsubsection{$\mathsf{FLGET}(
(\sshare{v_1},\sshare{p_1},\sshare{z_1},\sshare{s_1},\sshare{\err_1}),
(\sshare{v_2},\sshare{p_2},\sshare{z_2},\sshare{s_2},\sshare{\err_2}))$:}
\begin{enumerate}
  \item $\sshare{b} \asn \mathsf{FLLT}(
          (\sshare{v_1},\sshare{p_1},\sshare{z_1},\sshare{s_1},\sshare{\err_1}),
          (\sshare{v_2},\sshare{p_2},\sshare{z_2},\sshare{s_2},\sshare{\err_2}))$.
        %\item $\sshare{b_2} \asn \mathsf{FLEQ}(
        %        (\sshare{v_1},\sshare{p_1},\sshare{z_1},\sshare{s_1},\sshare{\err_1}),
        %        (\sshare{v_2},\sshare{p_2},\sshare{z_2},\sshare{s_2},\sshare{\err_2}))$.
  \item Return $1 - \sshare{b}$%$\mathsf{OR}(\sshare{b_1},\sshare{b_2})$.
\end{enumerate}

\msubsection{Conversion Routines}

\msubsubsection{$\mathsf{FLRound}((\sshare{v_1},\sshare{p_1},\sshare{z_1},\sshare{s_1},\sshare{\err_1},\mathsf{mode})$:}
  This, depending on $\mathsf{mode}$, computes either the floating point
  representation of the floor (if $\mathsf{mode}=0$) or the ceiling (if $\mathsf{mode}=1$)
  of the input floating point number.
  \begin{enumerate}
    \item $\sshare{a} \asn \mathsf{LTZ}(\sshare{p_1},k)$.
    \item $\sshare{b} \asn \mathsf{LT}(\sshare{p_1},-\ell+1,k)$.
    \item $\sshare{a \cdot b} \asn \sshare{a} \cdot \sshare{b}$.
    \item $\sshare{v_2},\sshare{2^{-p_1}} \asn \mathsf{Oblivious\_Trunc}(\sshare{v_1},\ell,(\sshare{a \cdot b}-\sshare{a}) \cdot \sshare{p_1})$.
            [Note, we save the computation of $\sshare{2^{-p_1}}$ which this routine computes when $0 \le p_1 < -\ell$, otherwise the returned share is of $2^0$.]
    \item $\sshare{c} \asn \mathsf{EQZ}(\sshare{v_2},\ell)$.
    \item $\sshare{v} \asn \sshare{v_1}-\sshare{v_2}+(1-\sshare{c}) \cdot \sshare{2^{-p_1}}
            \cdot \mathsf{XOR}(\mathsf{mode},\sshare{s_1})$.
    \item $\sshare{d} \asn \mathsf{EQ}(\sshare{v},2^\ell,\ell+1)$.
    \item $\sshare{v} \asn 2^{\ell-1} \cdot \sshare{d} + (1-\sshare{d}) \cdot \sshare{v}$.
    \item $\sshare{v} \asn (\sshare{a}-\sshare{a \cdot b}) \cdot \sshare{v}
            + \sshare{a \cdot b} \cdot (\mathsf{mode}-\sshare{s_1})
            + (1-\sshare{a}) \cdot \sshare{v_1}$.
    \item $\sshare{s} \asn (1-\sshare{b} \cdot \mathsf{mode}) \cdot \sshare{s_1}$.
    \item $\sshare{z} \asn \mathsf{OR}(\mathsf{EQZ}(\sshare{v},\ell),\sshare{z_1})$.
    \item $\sshare{v} \asn \sshare{v} \cdot (1-\sshare{z})$.
    \item $\sshare{p} \asn (\sshare{p_1}+\sshare{d} \cdot (\sshare{a}-\sshare{a \cdot b})) \cdot (1-\sshare{z})$.
    \item $\sshare{\err} \asn \sshare{\err_1}$.
    \item Return $(\sshare{v},\sshare{p},\sshare{z},\sshare{s},\sshare{\err})$.
  \end{enumerate}

  \paragraph{MAMBA Example:} To round a \verb|sfloat| value we could invoke the function as follows:
  \begin{lstlisting}[language={python}]
from Compiler import floatingpoint
x = sfloat(5.5)
mode = 0
# v, p, z, s, err are extracted from x
# retunrs the floor of x
y = floatingpoint.FLRound(x,mode)
\end{lstlisting}
  When \verb+mode=2+ then we get the ceil operation.


  \msubsubsection{$\mathsf{Int2Fx}(\sshare{a},k,f)$:}
  Given an integer $a \in \Zk$ this gives the equivalent integer $b$
  in $\Qk{f}$, namely $\overline{b}=a \cdot 2^f$.
  Note this means, to ensure correctness, that $|a|\le 2^{k-f}$.
  \begin{enumerate}
    \item Return $2^f \cdot \sshare{a}$.
  \end{enumerate}

  \paragraph{MAMBA Example:} To cast an \verb|int| or \verb|sint| register into a \verb|sfix| one, you could execute the following:
  \begin{lstlisting}[language={python}]
x = sfix(5.5)
# k, f are extracted from x
y = sfix.load_sint(x)
\end{lstlisting}

  \msubsubsection{$\mathsf{Int2FL}(\sshare{a},\gamma,\ell)$:}
  We assume $a \in \Zgam$, this could loose precision if $\gamma-1>\ell$.
  \begin{enumerate}
    \item $\lambda \asn \gamma-1$.
    \item $\sshare{s} \asn \mathsf{LTZ}(\sshare{a},\gamma)$.
    \item $\sshare{z} \asn \mathsf{EQZ}(\sshare{a},\gamma)$.
    \item $\sshare{a} \asn (1- 2 \cdot \sshare{s}) \cdot \sshare{a}$.
    \item $\sshare{a_{\lambda-1}},\ldots,\sshare{a_0}
            \asn \mathsf{BitDec}(\sshare{a},\lambda,\lambda)$.
    \item $\sshare{b_0},\ldots,\sshare{b_{\lambda-1}} \asn
            \mathsf{PreOp}(\mathsf{OR},\sshare{a_{\lambda-1}},\ldots,\sshare{a_0},\gamma)$.
    \item $\sshare{v} \asn \sshare{a} \cdot
            ( 1+\sum_{i=0}^{\lambda-1} 2^i \cdot (1-\sshare{b_i}))$.
    \item $\sshare{p} \asn - (\lambda-\sum_{i=0}^{\lambda-1} \sshare{b_i})$.
    \item If $(\gamma-1)>\ell$ then
          \begin{enumerate}
            \item $\sshare{v} \asn \mathsf{Trunc}(\sshare{v},\gamma-1,\gamma-\ell-1)$.
          \end{enumerate}
    \item Else
          \begin{enumerate}
            \item $\sshare{v} \asn 2^{\ell-\gamma+1} \cdot \sshare{v}$.
          \end{enumerate}
    \item $\sshare{p} \asn (\sshare{p}+\gamma-1-\ell) \cdot (1-\sshare{z})$.
    \item $\sshare{\err} \asn 0$.
    \item Return $(\sshare{v},\sshare{p},\sshare{z},\sshare{s},\sshare{\err})$.
  \end{enumerate}

  \paragraph{MAMBA Example:} To cast an \verb|int| or \verb|sint| register into a \verb|sfloat| one, you could execute the following:
  \begin{lstlisting}[language={python}]
x = sfloat(5.5)
# gamma and l are extracted from the system
y = sfloat(x)
\end{lstlisting}


  \msubsubsection{$\mathsf{Fx2Int}(\sshare{a},k,f)$:}
  Given a value $a \in \Qk{f}$ this gives the integer
$\floor{\overline{a}/2^f}$.
  \begin{enumerate}
    \item Return $\mathsf{Trunc}(\sshare{a},k,f)$.
  \end{enumerate}

  \paragraph{MAMBA Example:} To extract the integral component of a \verb|sfix| register, which is then encapsulated on a \verb|sint| register,
  and taking into account what is currently implemented, you could execute the following:
  \begin{lstlisting}[language={python}]
x = sifx(5.5)
# y stores 5 in a sint register
y = AdvInteger.Trunc(x.v, x.k, x.f, x.kappa, True)
\end{lstlisting}
  \msubsubsection{$\mathsf{FxFloor}(\sshare{a},k,f)$:}
  Given a value $a \in \Qk{f}$ this does the same, but
  gives the result as a fixed point value.
  \begin{enumerate}
    \item Return $2^f \cdot \mathsf{Trunc}(\sshare{a},k,f)$.
  \end{enumerate}
  \paragraph{MAMBA Example:} To floor an \verb|sfix| register, you could execute the following:
  \begin{lstlisting}[language={python}]
from Compiler import mpc_math
x = sifx(5.5)
# k and f are extracted from x
# y stores 5 in a sfix register
y = mpc_math.floor_fx(x)
\end{lstlisting}

  \msubsubsection{$\mathsf{Fx2FL}(\sshare{g},\gamma,f,\ell,k)$:}
  Converts $g \in \Qk{f}$ into a floating point number
  \begin{enumerate}
    \item ($\sshare{v},\sshare{p},\sshare{z},\sshare{s},\sshare{\err})
            \asn \mathsf{Int2FL}(\sshare{g},\gamma,\ell)$.
    \item $\sshare{p} \asn (\sshare{p}-f) \cdot (1-\sshare{z})$.
    \item Return $(\sshare{v},\sshare{p},\sshare{z},\sshare{s},\sshare{\err})$.
  \end{enumerate}
  \paragraph{MAMBA Example:} To cast from \verb|sfix| to \verb|sfloat|, you could execute the following:
  \begin{lstlisting}[language={python}]
# stores 5.5 on a sfloat register
x = sfloat(sfix(5.5))
\end{lstlisting}

  \iffalse
    % this seems not to be used elsewhere
    \msubsubsection{$\mathsf{FL2Int}((\sshare{v},\sshare{p},\sshare{z},\sshare{s},\sshare{\err}),\ell,k,\gamma)$:}
      Note the two calls to $\mathsf{Mod2m}$ below can be combined into
      one, as can the two calls to $\mathsf{Pow2}$.
      The output here is in $\Zgam$.
    \begin{enumerate}
      \item $\sshare{v'},\sshare{p'},\sshare{z'},\sshare{s'},\sshare{\err'}
              \asn \mathsf{FLRound}(\sshare{v},\sshare{p},\sshare{z},\sshare{s},\sshare{\err},\mathsf{2})$.
      \item $\sshare{a} \asn \mathsf{LT}(\sshare{p'},\gamma-1,k)$.
      \item $\sshare{b} \asn \mathsf{LT}(\gamma-\ell-1,\sshare{p'},k)$.
      \item $\sshare{c} \asn \mathsf{LTZ}(\sshare{p'},k)$.
      \item $\sshare{b \cdot c} \asn \sshare{b} \cdot \sshare{c}$.
      \item $\sshare{m} \asn \sshare{a}\cdot (\sshare{b} -\sshare{b \cdot c})
              \cdot (\gamma-1-\sshare{p'})$.
      \item $\sshare{u} \asn \mathsf{Mod2m}(\sshare{v'},\ell,\sshare{m})$.
      \item $\sshare{v'} \asn (\sshare{b} -\sshare{b \cdot c}) \cdot ( \sshare{u}-\sshare{v'}) +\sshare{v'}$
      \item $\sshare{2^{-p'}} \asn \mathsf{Pow2}(-\sshare{c} \cdot \sshare{p'},\ell)$.
      \item $\sshare{2^{p'}} \asn \mathsf{Inv}(\sshare{2^{-p'}})$.
      \item $\sshare{v'} \asn (\sshare{c} \cdot \sshare{2^{p'}}+1-\sshare{c})
              \cdot \sshare{v'}$.
      \item $\sshare{w} \asn \mathsf{Mod2m}(\sshare{v'},\ell,
              \sshare{b \cdot c}\cdot(\gamma-1))$.
      \item $\sshare{v'} \asn \sshare{b \cdot c} \cdot (\sshare{w}-\sshare{v'}) +\sshare{v'}$.
      \item $\sshare{2^{p'}} \asn \mathsf{Pow2}(\sshare{a} \cdot (1-\sshare{c})
              \cdot \sshare{p'},\gamma-1)$.
      \item $\sshare{g} \asn (1-\sshare{z'}) \cdot (1-2 \cdot \sshare{s'})
              \cdot \sshare{2^{p'}} \cdot \sshare{a}
              \cdot \sshare{v'}$.
      \item Return $\sshare{g}$.
    \end{enumerate}
    \paragraph{MAMBA Example:} To extract the integral component of a \verb|sfloat| input,
    given the current implementation, you could execute the following:
    \begin{lstlisting}[language={python}]
from Compiler import floatingpoint
# v, p, z, s, err are extracted from x and l, k, gamma are system parameters
x = sfloat(5.5)
# retunrs the floor of x
y = floatingpoint.FLRound(x,0)
\end{lstlisting}
  \fi

  \msubsubsection{$\mathsf{FL2Fx}((\sshare{v},\sshare{p},\sshare{z},\sshare{s},\sshare{\err}),\ell,k,\gamma,f)$:}
\begin{enumerate}
  \item $\sshare{b} \asn \mathsf{LT}(\sshare{p},2^{k-1}-f,k)$.
  \item $\sshare{g} \asn \mathsf{FL2Int}((\sshare{v},\sshare{p}+f,\sshare{z},
          \sshare{s},\sshare{\err}),\ell,k,\gamma)$.
  \item Return $\sshare{g} \cdot \sshare{b}$.
\end{enumerate}
\paragraph{MAMBA Example:} To cast an \verb|sfloat| register into a \verb|sfix| one, you could execute the following:
\begin{lstlisting}[language={python}]
# stores 5.5 on a sfix register
# v, p, z, s, err are extracted from x and l, k, gamma are system parameters
x = sfix(sfloat(5.5))
\end{lstlisting}

\msubsection{SQRT Functions}
\todo{These functions are only currently supported in their fixed point versions in Mamba,
all variants are however supported in the Rust pipeline.}
The original description of the protocols for the fixed point square root algorithm are included in:
\begin{itemize}
  \item Secure Distributed Computation of the Square Root and Applications, {\em ISPEC 2012}
        \cite{Liedel12}.
\end{itemize}
The floating point variant is in
\begin{itemize}
  \item Secure Computation on Floating Point Numbers {\em NDSS 2013} \cite{ABZS13}.
\end{itemize}
Additionally, we provide a simplified implementation for the Square Root on fixed point variables, that is appropriate for inputs of any size. We make use Liedel's protocol, when the input's size makes it possible, as we explain later in this section.
Both algorithms makes use of two constants
\[ \alpha = -0.8099868542, \quad \beta  =  1.787727479.  \]
These are the solutions of the system of equations
\begin{align*}
  E(x)                       & = \frac{\alpha \cdot x+\beta-\frac{1}{\sqrt{x}}}{\frac{1}{\sqrt{x}}}, \\
  M                          & = \frac{\sqrt{3}}{3} \cdot \sqrt{\frac{-\beta}{\alpha}}
  \cdot \left(\frac{2}{3} \cdot\beta
  -\frac{\sqrt{3}}{\sqrt{\frac{-\beta}{\alpha}}}
  \right),                                                                                           \\
  E\left( \frac{1}{2}\right) & = E(1) = -M.
\end{align*}

\msubsubsection{$\mathsf{ParamFxSqrt}(\sshare{x},k,f)$:}
This algorithm uses the sub-algorithm $\mathsf{LinAppSQ}$ defined below, note that \verb|LinAppSQ| returns an scaled $1/\sqrt{x} \cdot 2^{f}$.
The algorithm only works when $3 \cdot k -2 \cdot f$ is less than the system precision, which is by default equal to $20$.
In a future release we will extend the sqrt function to cope with other input ranges.
\begin{enumerate}
  \item $\theta \asn \ceil{\log_2 (k/5.4)}$.
  \item $\sshare{y_0} \asn \mathsf{LinAppSQ}(\sshare{x},k,f)$.
  \item $\sshare{y_0} \asn \sshare{y_0} \cdot 1/2^{f}$.
  \item $\sshare{g_0} \asn \sshare{y_0} \cdot \sshare{x}$.
  \item $\sshare{g_0} \asn \sshare{y_0} \cdot 1/2^{f}$.
        %\item $\sshare{g_0} \asn \mathsf{TruncPr}(\sshare{g_0},k,f)$.
        %\item $\sshare{h_0} \asn \mathsf{FxDiv}(\sshare{g_0},2,k,f)$.
  \item $\sshare{g_0} \asn \sshare{y_0} \cdot 1/2$.
  \item $\sshare{gh_0} \asn \sshare{g_0} \cdot \sshare{h_0}$.
        %\item $\sshare{gh} \asn \mathsf{TruncPr}(\sshare{gh},k,f)$.
  \item $\sshare{g} \asn \sshare{g_0}$.
  \item $\sshare{h} \asn \sshare{h_0}$.
  \item $\sshare{gh} \asn \sshare{gh_0}$.
  \item For $i \in[1,\ldots,\theta-2]$ do
        \begin{enumerate}
          \item $\sshare{r} \asn 3/2-\sshare{gh}$.
          \item $\sshare{g} \asn \sshare{g} \cdot \sshare{r}$.
          \item $\sshare{h} \asn \sshare{h} \cdot \sshare{r}$.
                %\item $\sshare{g} \asn \mathsf{TruncPr}(\sshare{g},k,f)$.
                %\item $\sshare{h} \asn sf{TruncPr}(\sshare{h},k,f)$.
          \item $\sshare{gh} \asn \sshare{g} \cdot \sshare{h}$.
                %\item $\sshare{gh} \asn \mathsf{TruncPr}(\sshare{gh},k,f)$.
        \end{enumerate}
  \item $\sshare{r} \asn 3/2-\sshare{gh}$.
  \item $\sshare{h} \asn \sshare{h} \cdot \sshare{r}$.
        %\item $\sshare{h} \asn \mathsf{TruncPr}(\sshare{h},k,f)$.
  \item $\sshare{H} \asn 4 \cdot (\sshare{h}^2)$.
  \item $\sshare{H} \asn \sshare{H}\cdot \sshare{x}$.
  \item $\sshare{H} \asn (3) - \sshare{H}$.
  \item $\sshare{H} \asn \sshare{h} \cdot \sshare{H}$.
  \item $\sshare{g} \asn \sshare{H} \cdot \sshare{x}$.
        %\item $\sshare{g} \asn \mathsf{FxDiv}(\sshare{g},2,k,f)$.
        %\item $\sshare{g} \asn \mathsf{TruncPr}(\sshare{g},4 \cdot k,4 \cdot f)$.
  \item Return $\sshare{g}$.
\end{enumerate}

\msubsubsection{$\mathsf{SimplifiedFxSqrt}(\sshare{x}, k, f)$:}
This algorithm uses the sub-algorithm $\mathsf{NormSQ}$ defined above. Among the values it returns, we base our approximation by directly using $w =2^{m/2}$. From that point it approximates the  value of $\sqrt{x}$ by calculating $\frac{x}{2^{m/2}}$. To avoid any loss of precision, we reuse \verb|sfix| instantiation process.
The algorithm work on the precision of the system and can solve values on any range. Its behaviour is still experimental. The function is designed in such a way that there is no restriction on the size $f$.
\begin{enumerate}
  \item $\theta \asn \max{(\ceil{\log_2(k)},6)}$.
  \item $\sshare{m_{odd}}, \sshare{w} \asn \mathsf{SimplifiedNormSQ}(\sshare{x},k)$.
  \item $\sshare{m_{odd}} \asn (1 - 2 \cdot \sshare{m_{odd}}) \cdot (f \quad \% \quad 2)$.
  \item $\sshare{w} \asn (2 \cdot \sshare{w} -\sshare{w}) \cdot (1 - \sshare{m_{odd}}) \cdot (f \quad \% \quad 2) + \sshare{w}$.
  \item $\sshare{w} \asn sfix( \sshare{w} \cdot 2 ^{\frac{f - f \quad \% \quad 2}{2}})$.
  \item $\sshare{w} \asn (\sqrt{2}\cdot \sshare{w} - \sshare{w})\cdot \sshare{m_{odd}} + \sshare{w}$.
  \item $\sshare{y_0} \asn \frac{1}{\sshare{w}} $.
  \item $\sshare{g_0} \asn \sshare{y_0} \cdot \sshare{x}$.
  \item $\sshare{g_0} \asn \sshare{y_0} \cdot 1/2$.
  \item $\sshare{gh_0} \asn \sshare{g_0} \cdot \sshare{h_0}$.
  \item $\sshare{g} \asn \sshare{g_0}$.
  \item $\sshare{h} \asn \sshare{h_0}$.
  \item $\sshare{gh} \asn \sshare{gh_0}$.
  \item For $i \in[1,\ldots,\theta-2]$ do
        \begin{enumerate}
          \item $\sshare{r} \asn 3/2-\sshare{gh}$.
          \item $\sshare{g} \asn \sshare{g} \cdot \sshare{r}$.
          \item $\sshare{h} \asn \sshare{h} \cdot \sshare{r}$.
                %\item $\sshare{g} \asn \mathsf{TruncPr}(\sshare{g},k,f)$.
                %\item $\sshare{h} \asn sf{TruncPr}(\sshare{h},k,f)$.
          \item $\sshare{gh} \asn \sshare{g} \cdot \sshare{h}$.
                %\item $\sshare{gh} \asn \mathsf{TruncPr}(\sshare{gh},k,f)$.
        \end{enumerate}
  \item $\sshare{r} \asn 3/2-\sshare{gh}$.
  \item $\sshare{h} \asn \sshare{h} \cdot \sshare{r}$.
        %\item $\sshare{h} \asn \mathsf{TruncPr}(\sshare{h},k,f)$.
  \item $\sshare{H} \asn 4 \cdot (\sshare{h}^2)$.
  \item $\sshare{H} \asn \sshare{H}\cdot \sshare{x}$.
  \item $\sshare{H} \asn (3) - \sshare{H}$.
  \item $\sshare{H} \asn \sshare{h} \cdot \sshare{H}$.
  \item $\sshare{g} \asn \sshare{H} \cdot \sshare{x}$.
        %\item $\sshare{g} \asn \mathsf{FxDiv}(\sshare{g},2,k,f)$.
        %\item $\sshare{g} \asn \mathsf{TruncPr}(\sshare{g},4 \cdot k,4 \cdot f)$.
  \item Return $\sshare{g}$.
\end{enumerate}

\msubsubsection{$\mathsf{FxSqrt}(\sshare{x}, k \asn \mathtt{sfix.k}, f \asn \mathtt{sfix.f})$:}
Our \verb|FxSqrt| functionality returns the square root of any fixed point input. It receives an input value $\sshare{x}$, from which it calculates the square root, and optional parameters regarding its bit-length and bit-wise precision. The functionality is going to make use of our \verb|SimplifiedFxSqrt| process by default, and the somewhat more efficient Liedel's method instead when the $3 \cdot k - 2 \cdot f < \mathtt{sfix.f}$ bound, provided by his paper, is met.

\begin{enumerate}
  \item if $(3 \cdot k - 2 \cdot f >= \mathtt{sfix.f})$:
        \begin{enumerate}
          \item Return $\mathsf{SimplifiedFxSqrt}(\sshare{x}, k, f)$.
        \end{enumerate}
  \item else:
        \begin{enumerate}
		\item $\sshare{x} \asn \mathsf{Trunc}(\sshare{x} \cdot 2^{f}, sfix.k, sfix.k-sfix.f)$
          \item Return $\mathsf{ParamFxSqrt}(\sshare{x}, k, f)$.
        \end{enumerate}
\end{enumerate}

\paragraph{MAMBA Example:} To obtain the \verb|sqrt| of any value, you could execute the following:
\begin{lstlisting}[language={python}]
from Compiler import mpc_math
k = 5
f = 2

x = sfix(6.25)
y = sfix(144)
z = sfix (257.5)

# returns the sqrt of the number, i.e. 2.5
# inputs have to be expressed such that:
# x * 2^f \in Z_q
# and 3*k -2*f < sfix.f (system precision)
# by default system precision is 20 bits.
a = mpc_math.sqrt(x, k, f)

# when you don't specify k and f, the system uses
# the default sfix values, and hence the simplified
# version for any value range, at the cost of an
# additional division call.

b = mpc_math.sqrt(y)
c = mpc_math.sqrt(z)

\end{lstlisting}

\msubsubsection{$\mathsf{LinAppSQ}(\sshare{b},k,f)$:}
We based this section on the contents of the original paper, \cite{Liedel12}.
However we corrected the typos from the original work, the result is as follows:
\begin{enumerate}
  \item $\alpha \asn (-0.8099868542) \cdot 2^k$.
  \item $\beta \asn (1.787727479) \cdot 2^{2\cdot k}$.
  \item $(\sshare{c},\sshare{v},\sshare{m},\sshare{W}) \asn \mathsf{NormSQ}(\sshare{b},k,f)$.
  \item $\sshare{w} \asn \alpha \cdot \sshare{c}+\beta$.
  \item $\sshare{m} \asn \mathsf{Mod2}(\sshare{m},\ceil{\log_2 k})$.
  \item $\sshare{w} \asn \sshare{w} \cdot \sshare{W} \cdot \sshare{v}$.
  \item $\sshare{w} \asn \mathsf{FxDiv}(\sshare{w},2^{f/2},w.k,w.f)$.
  \item $\sshare{w} \asn \mathsf{FxDiv}(\sshare{w},2^{3 \cdot k - 2 \cdot f},w.k,w.f)$.
        %\item $\sshare{w} \asn \mathsf{TruncPr}(\sshare{w},3\cdot k, 3 \cdot k - 2 \cdot f)$.
  \item $\sshare{w} \asn (1-\sshare{m}) \cdot \sshare{w} \cdot 2^f
          +(\sqrt{2} \cdot 2^f) \cdot \sshare{m}\cdot \sshare{w}$.
  \item Return $\sshare{w}$ % \asn \mathsf{TruncPr}(\sshare{w},k,f)$.
\end{enumerate}


\msubsubsection{$\mathsf{FLSqrt}((\sshare{v_1},\sshare{p_1},\sshare{z_1},\sshare{s_1},\sshare{\err_1}))$:}
  Below we let $\ell_0$ denote the lsb of $\ell$,
$(v_\alpha,p_\alpha,z_\alpha,s_\alpha)$
  (resp.  $(v_\beta,p_\beta,z_\beta,s_\beta)$)
  denote the floating point representation of the constant $\alpha$ (resp. $\beta$)
  given above,
  and $v_{\sqrt{2}}$ and $p_{\sqrt{2}}$ represent the $\ell$-bit significand and exponent of
$\sqrt{2}$ in floating point representation.
  \begin{enumerate}
    \item $\sshare{b} \asn \mathsf{BitDec}(\sshare{p_1},\ell,1)$.
    \item $\sshare{c} \asn \mathsf{XOR}(\sshare{b},\ell_0)$.
    \item $\sshare{p} \asn 2^{-1} \cdot (\sshare{p_1}-\sshare{b})
            +\floor{\ell/2}+\mathsf{OR}(\sshare{b},\ell_0)$.
    \item $(\sshare{v_2},\sshare{p_2},\sshare{z_2},\sshare{s_2},\sshare{\err_2})
            \asn \mathsf{FLMult}((\sshare{v_1},-\ell,0,0,0),~~
		  (v_\alpha,p_\alpha,z_\alpha,s_\alpha,0))$.
    \item $(\sshare{v_0},\sshare{p_0},\sshare{z_0},\sshare{s_0},\sshare{\err_0})
            \asn \mathsf{FLAdd}((\sshare{v_2},\sshare{p_2},\sshare{z_2},\sshare{s_2},\sshare{\err_2}),~~
		  (v_\beta,p_\beta,z_\beta,s_\beta,0))$.
    \item $(\sshare{v_g},\sshare{p_g},\sshare{z_g},\sshare{s_g},\sshare{\err_g})
            \asn \mathsf{FLMult}((\sshare{v_1},-\ell,0,0,0),(\sshare{v_0},~~
		  \sshare{p_0},\sshare{z_0},\sshare{s_0},\sshare{\err_0}))$.
    \item  $(\sshare{v_h},\sshare{p_h},\sshare{z_h},\sshare{s_h},\sshare{\err_h})
            \asn   (\sshare{v_0},\sshare{p_0}-1,\sshare{z_0},\sshare{s_0},\sshare{\err_0})$.
    \item For $i \in [1,\ldots,\ceil{\log_2(\ell/5.4)}-1]$ do
          \begin{enumerate}
            \item $(\sshare{v_2},\sshare{p_2},\sshare{z_2},\sshare{s_2},\sshare{\err_2})
                    \asn \mathsf{FLMult}(
                    (\sshare{v_g},\sshare{p_g},\sshare{z_g},\sshare{s_g},\sshare{\err_g}),~~
                    (\sshare{v_h},\sshare{p_h},\sshare{z_h},\sshare{s_h},\sshare{\err_h})
                    )$.
            \item $(\sshare{v_2},\sshare{p_2},\sshare{z_2},\sshare{s_2},\sshare{\err_2})
                    \asn \mathsf{FLSub}(
                    (3 \cdot 2^{\ell-2},-(\ell-1),0,0,0),~~
                    (\sshare{v_2},\sshare{p_2},\sshare{z_2},\sshare{s_2},\sshare{\err_2})
                    )$.
            \item $(\sshare{v_g},\sshare{p_g},\sshare{z_g},\sshare{s_g},\sshare{\err_g})
                    \asn \mathsf{FLMult}(
                    (\sshare{v_g},\sshare{p_g},\sshare{z_g},\sshare{s_g},\sshare{\err_g}),~~
                    (\sshare{v_2},\sshare{p_2},\sshare{z_2},\sshare{s_2},\sshare{\err_2})
                    )$.
            \item $(\sshare{v_h},\sshare{p_h},\sshare{z_h},\sshare{s_h},\sshare{\err_h})
                    \asn \mathsf{FLMult}(
                    (\sshare{v_h},\sshare{p_h},\sshare{z_h},\sshare{s_h},\sshare{\err_h}),~~
                    (\sshare{v_2},\sshare{p_2},\sshare{z_2},\sshare{s_2},\sshare{\err_2})
                    )$.
          \end{enumerate}
    \item $(\sshare{v_{h^2}},\sshare{p_{h^2}},\sshare{z_{h^2}},\sshare{s_{h^2}},\sshare{\err_{h^2}})
            \asn \mathsf{FLMult}(
            (\sshare{v_h},\sshare{p_h},\sshare{z_h},\sshare{s_h},\sshare{\err_h}),~~
            (\sshare{v_h},\sshare{p_h},\sshare{z_h},\sshare{s_h},\sshare{\err_h})
            )$.
    \item $(\sshare{v_2},\sshare{p_2},\sshare{z_2},\sshare{s_2},\sshare{\err_2})
            \asn \mathsf{FLMult}(
            (\sshare{v_1},-\ell,0,0,0),~~
            (\sshare{v_{h^2}},\sshare{p_{h^2}},\sshare{z_{h^2}},\sshare{s_{h^2}},\sshare{\err_{h^2}})$.
    \item $(\sshare{v_2},\sshare{p_2},\sshare{z_2},\sshare{s_2},\sshare{\err_2})
            \asn \mathsf{FLSub}(
            (3 \cdot 2^{\ell-2},-(\ell-1),0,0,0),~~
            (\sshare{v_2},\sshare{p_2}+1,\sshare{z_2},\sshare{s_2},\sshare{\err_2})
            )$.
    \item $(\sshare{v_h},\sshare{p_h},\sshare{z_h},\sshare{s_h},\sshare{\err_h})
            \asn \mathsf{FLMult}(
            (\sshare{v_h},\sshare{p_h},\sshare{z_h},\sshare{s_h},\sshare{\err_h}),~~
            (\sshare{v_2},\sshare{p_2},\sshare{z_2},\sshare{s_2},\sshare{\err_2})
            )$.
    \item $(\sshare{v_2},\sshare{p_2},\sshare{z_2},\sshare{s_2},\sshare{\err_2})
            \asn \mathsf{FLMult}(
            (\sshare{v_1},-\ell,0,0,0),~~
            (\sshare{v_h},\sshare{p_h}+1,\sshare{z_h},\sshare{s_h},\sshare{\err_h})
            )$.
    \item $(\sshare{v_2},\sshare{p_2},\sshare{z_2},\sshare{s_2},\sshare{\err_2})
            \asn \mathsf{FLMult}(
            (\sshare{v_2},\sshare{p_2},\sshare{z_2},\sshare{s_2},\sshare{\err_2}),~~
            (2^{\ell-1} \cdot (1-\sshare{c}) + v_{\sqrt{2}} \cdot \sshare{c},
	    -(1-\sshare{c}) \cdot (\ell-1)+p_{\sqrt{2}} \cdot \sshare{c},0,0,0)
	    )$.
    \item $\sshare{p} \asn (\sshare{p_2}+\sshare{p}) \cdot (1-\sshare{z_1})$.
    \item $\sshare{v} \asn \sshare{v_2} \cdot (1-\sshare{z_1})$.
    \item $\sshare{\err} \asn \sshare{\err_2} + \sshare{s_1}$.
    \item Return $(\sshare{v},\sshare{p},\sshare{z_1},\sshare{s_1},\sshare{\err})$.
  \end{enumerate}

  \msubsection{EXP and LOG Functions}
  \todo{The exp and log functions are only currently supported in their fixed point versions
  in the Mamba pipeline. Both functions are supported in the Rust pipeline.}

  A secure fixed point exponentiation and logarithm algorithm is not found anywhere, so
  this is our own one derived from the identities in the book.
  \begin{itemize}
    \item {\em Computer Approximations} by Hart from 1968 \cite{Hart:1978:CA:540084}.
  \end{itemize}
  The floating point variants are in
  \begin{itemize}
    \item Secure Computation on Floating Point Numbers {\em NDSS 2013} \cite{ABZS13}.
  \end{itemize}
  Once we have defined $\mathsf{FxExp2}$ and $\mathsf{FxLog2}$
  (resp. $\mathsf{FLExp2}$ and $\mathsf{FLLog2}$) we can
  define the following functions from the usual identities
  for non-secret values of the base $b$:
  \begin{align*}
    \log_b x & = (\log_b 2) \cdot \mathsf{Log2}(x),       \\
    x^y      & = \mathsf{Exp2}(y \cdot \mathsf{Log2}(x)), \\
    \exp   x & = \mathsf{Exp2}(x \cdot \log_2 e),         \\
  \end{align*}
  Note, that the functions on these section require specific \verb|sfloat| parametrization, in accordance to the algorithms in this section. They support secret shared \verb|sfix| $x$ ad $y$, as well as public floating point or integer inputs.  We can define these operations as follows:

  \msubsubsection{$\mathsf{FxExp2}(\sshare{a},k,f)$:}
  This algorithm computes $2^a$ as a fixed point calculation.
  First takes the integer and fractional part of the input
$|a/2^f|$, which we denote by $b$ and $c$.
  We then compute $d=2^b$, which will clearly overflow
  if $b>k-f$, but we ignore this error (if the user is stupid
  enough to put in garbage, they get garbage out).
  We then compute $e=2^c$, as $0 \le c \le 1$ via the
  polynomial the following polynomial\footnote{Note polynomial
    $P_{1045}(X)$ from Hart \cite{Hart:1978:CA:540084} is incorrect
    and does not give an accurate result}.
  %with coefficients
  %\begin{center}
  %\begin{tabular}{|c|c|l|}
  %\hline
  %0 & 1  & +.10000 00077 44302 1686 \\
  %1 & 0  & +.69314 71804 26163 82779 5756 \\
  %2 & 0  & +.24022 65107 10170 64605 384 \\
  %3 & -1 & +.55504 06862 04663 79157 744 \\
  %4 & -2 & +.96183 41225 88046 23749 77 \\
  %5 & -2 & +.13327 30359 28143 78193 29 \\
  %6 & -3 & +.15510 74605 90052 57397 8 \\
  %7 & -4 & +.14197 84739 97656 06711 \\
  %8 & -5 & +.18633 47724 13796 7076 \\
  %\hline
  %\end{tabular}
  %\end{center}
  %which gives a relative error of at most $10^{-12.11}$
  %if computed exactly.
  %The table should be read as line $(i,a,b)$ giving
  %the $i$th coefficient of the polynomial being
  %$b \cdot 10^a$.
  Our polynomial (which we produced using a Chebyshev approximation)
  is of degree nine and has coefficients given by
  \begin{center}
    \begin{tabular}{|c|l|}
      \hline
      0 & 0.99999999999998151058451       \\
      1 & 0.69314718056364205693851       \\
      2 & 0.24022650683729748257646       \\
      3 & 0.0555041102193305250618        \\
      4 & 0.0096181190501642860210497     \\
      5 & 0.0013333931011014250476911     \\
      6 & 0.00015395144945146697380844    \\
      7 & 0.000015368748541192116946474   \\
      8 & 0.0000012256971722926501833228  \\
      9 & 0.00000014433329807023165258784 \\
      \hline
    \end{tabular}
  \end{center}
  Given $d$ and $e$ one can now compute
$2^{|a|}=2^{b+c}=2^b \cdot 2^c=d \cdot e$,
  and the final dealing with the sign of $a$ can
  be done by an inversion.
  We denote by $\mathsf{FxPol}(P_{1045},\sshare{x},k,f)$ the evaluation
  of the polynomial $P_{1045}$ on the fixed point input $\sshare{x}$
  where $x \in \Qk{f}$. This is done by Horner's rule.
  \begin{enumerate}
    \item $\sshare{s}=\mathsf{FxLTZ}(\sshare{a})$.
    \item $\sshare{a} \asn (1-2 \cdot \sshare{s}) \cdot \sshare{a}$.
    \item $\sshare{b} \asn \mathsf{Fx2Int}(\sshare{a},k,f)$.
    \item $\sshare{c} \asn \sshare{a}-\mathsf{Int2Fx}(\sshare{b},k,f)$.
    \item $\sshare{d} \asn \mathsf{Int2Fx}(\mathsf{Pow2}(\sshare{b},k),k,f)$. [This will produce an invalid result if $b$ is too big, in which case the result cannot be held in an Fx in any case]
    \item $\sshare{e} \asn \mathsf{FxPol}(P_{1045},\sshare{c},k,f)$.
    \item $\sshare{g} \asn \mathsf{FxMult}(\sshare{d},\sshare{e},k,f)$.
    \item $\sshare{g^{-1}} \asn \mathsf{FxDiv}(2^f,\sshare{g},k,f)$.
    \item $\sshare{a} \asn (1-\sshare{s}) \cdot g+ \sshare{s} \cdot \sshare{g^{-1}}$.
    \item Return $\sshare{a}$.
  \end{enumerate}
  The above works, but we have found a little numerical
  instability due to the division operation.
  \paragraph{MAMBA Example:} To obtain $2^y$ where $y$ is secret shared you could run the following:
  \begin{lstlisting}[language={python}]
from Compiler import mpc_math

y =sfix(4)

# returns 2^4
# extracts k and f from y
exp2_y=mpc_math.exp2_fx(sfix(y))
\end{lstlisting}

  \msubsubsection{$\mathsf{FLExp2}((\sshare{v_1},\sshare{p_1},\sshare{z_1},\sshare{s_1},\sshare{\err_1}))$:}
This method assumes that $k \le \ell$.
We do not support a method if $k > \ell$, and so if this happens
we will signal an error.
\begin{enumerate}
  \item If $k>\ell$ then $\err_1 \asn 1$.
  \item $\max \asn \ceil{\log_2(2^{k-1}-1+\ell)-\ell+1}$.
  \item $\sshare{a} \asn \mathsf{LT}(\sshare{p_1},\max,k)$.
  \item $\sshare{b} \asn \mathsf{LT}(\sshare{p_1},-\ell+1,k)$.
  \item $\sshare{c} \asn \mathsf{LT}(\sshare{p_1},-2 \cdot \ell+1,k)$.
  \item $\sshare{(1-c) \cdot a} \asn (1-\sshare{c}) \cdot \sshare{a}$.
  \item $\sshare{p_2} \asn -\sshare{(1-c) \cdot a} \cdot (\sshare{b} \cdot \ell +\sshare{p_1})$.
  \item $\sshare{x},\sshare{2^{p_2}} \asn \mathsf{Trunc}(\sshare{v_1},\ell,\sshare{p_2})$.
  \item $\sshare{y} \asn \sshare{v_1}-\sshare{x} \cdot \sshare{2^{p_2}}$.
  \item $\sshare{d} \asn \mathsf{EQZ}(\sshare{y},\ell)$.
  \item $\sshare{b \cdot s_1} \asn \sshare{b} \cdot \sshare{s_1}$.
  \item $\sshare{(1-d) \cdot s_1} \asn (1-\sshare{d}) \cdot \sshare{s_1}$.
  \item $\sshare{x} \asn (1-\sshare{b \cdot s_1})
          \cdot (\sshare{x}-\sshare{(1-d)\cdot s_1})
          +\sshare{b \cdot s_1} \cdot(2^\ell-1+\sshare{d}-\sshare{x})$.
  \item $\sshare{y} \asn \sshare{(1-d) \cdot s_1} \cdot (\sshare{2^{p_2}}-\sshare{y})
          + (1-\sshare{s_1})\cdot \sshare{y}$.
  \item $\sshare{w} \asn \sshare{(1-c) \cdot a} \cdot
          ((1-\sshare{b}) \cdot\sshare{x}+\sshare{b \cdot s_1})
          \cdot (1-2\cdot \sshare{s_1})-\sshare{c} \cdot \sshare{s_1}$.
  \item $\sshare{u} \asn \sshare{(1-c) \cdot a}
          \cdot (\sshare{b} \cdot \sshare{x}
          +(1-\sshare{b}) \cdot 2^\ell \cdot \mathsf{Inv}(\sshare{2^{p_2}})
          \cdot \sshare{y})
          +(2^\ell-1)\cdot \sshare{c} \sshare{s_1}$.
  \item $\sshare{u_\ell},\ldots,\sshare{u_1} \asn \mathsf{BitDec}(\sshare{u},\ell,\ell)$.
  \item For $i \in [1,\ldots,\ell]$ do
        \begin{enumerate}
          \item~ [In this loop $(cv_i,cp_i,0,0)$ represents the floating point number $2^{2^{-i}}$].
          \item $\sshare{a_i} \asn 2^{\ell-1} \cdot (1-\sshare{u_i})+cv_i \cdot \sshare{u_i}$.
          \item $\sshare{b_i} \asn -(\ell-1) \cdot (1-\sshare{u_i})+cp_i \cdot \sshare{u_i}$.
        \end{enumerate}
  \item $(\sshare{v_u},\sshare{p_i},0,0)
          \asn \mathsf{FLProd}((\sshare{a_1},\sshare{b_1},0,0), \ldots,
          (\sshare{a_\ell},\sshare{b_\ell},0,0))$.
          [This implements a product of $\ell$ floating point values, which is
            performed via a binary tree style method.]
  \item $\sshare{p} \asn \sshare{a} \cdot (\sshare{w}+\sshare{p_u})
          +2^{k-1}\cdot (1-\sshare{a}) \cdot (1-2 \sshare{s_1})$.
  \item $\sshare{v} \asn 2^{\ell-1} \cdot \sshare{z_1}+(1-\sshare{z_1}) \cdot \sshare{v_u}$.
  \item $\sshare{p} \asn -\sshare{z_1} \cdot (\ell-1)
          +(1-\sshare{z_1}) \cdot \sshare{p}$.
  \item $\sshare{\err} \asn \sshare{\err_1} +  \mathsf{FlowDetect}(\sshare{p})$
  \item Return $(\sshare{v},\sshare{p},0,0,\sshare{\err})$.
\end{enumerate}

\msubsubsection{$\mathsf{FxLog2}(\sshare{a},k,f)$:}
We first map $a$ to a value $v$ in the interval $[1/2,1]$ by essentially
converting to a floating point number.
So we a have $a=(v/2^k) \cdot 2^p$ where $v, p \in \Zk$,
and $v/2^k \in [1/2,1]$.
Thus we have
$\log_2 a = p+\log_2 (v/2^k)$, and we then treat $v$ as a fixed
point number and apply the Pade approximation $P_{2524}/Q_{2524}$
from Hart's book \cite{Hart:1978:CA:540084},
which produces an {\em absolute} error of $10^{-8.32}$.
We denote by $\mathsf{FxPade}(P_{2524},Q_{2524},\sshare{x},k,f)$ the evaluation
of the rational function $P_{2524}/Q_{2524}$ on the fixed point input $\sshare{x}$
where $x \in \Qk{f}$.
The Pade approximation is given be the rational function defined
by the following table
\begin{center}
  \begin{tabular}{|c|c|c|l|}
    \hline
    P & 0 & 1 & -.20546 66719 51 \\
    P & 1 & 1 & -.88626 59939 1  \\
    P & 2 & 1 & +.61058 51990 15 \\
    P & 3 & 1 & +.48114 74609 89 \\
    Q & 0 & 0 & +.35355 34252 77 \\
    Q & 1 & 1 & +.45451 70876 29 \\
    Q & 2 & 1 & +.64278 42090 29 \\
    Q & 3 & 1 & +.1              \\
    \hline
  \end{tabular}
\end{center}

\begin{enumerate}
  \item $(\sshare{v},\sshare{p},\sshare{z},\sshare{s},\sshare{\err})
          \asn \mathsf{Fx2FL}(\mathsf{a},k,f,k,k)$.
  \item $\sshare{a} \asn \mathsf{FxPade}(P_{2524},Q_{2524},\sshare{v},k,k)$.
  \item $\sshare{a} \asn \sshare{a}+\sshare{p} + f$.
  \item $\sshare{a} \asn \sshare{a} \cdot (1-\sshare{z}) \cdot (1-\sshare{s}) \cdot (1-\cdot \sshare{\err})$.
  \item Return $\sshare{a}$.
\end{enumerate}

\paragraph{MAMBA Example:} To obtain \verb|log2|$(x)$ where $y$ is secret shared you could run the following:
\begin{lstlisting}[language={python}]
from Compiler import mpc_math
sfloat.vlen = 15   # Length of mantissa in bits
sfloat.plen = 10   # Length of exponent in bits
sfloat.kappa = 40  # Statistical security parameter for floats
sfix.set_precision(20,40)
cfix.set_precision(20,40)

x =sfix(4)
# extracts k and f from y
# returns log_2(4)
log2_x=mpc_math.log2_fx(sfix(x))
\end{lstlisting}
Note, internally \verb+log2_fx+ uses a polynomial to approximate the logarithm, but it uses the \verb+sfloat+ arithmetic to do this.
Thus, you need to set up \verb+sfloat+ and \verb+sfix+ correctly for this to work.
In particular if you look deep inside the code there is the following cryptic remark in relation to \verb+log2_fx+...
\begin{verbatim}
     # Note that sfloat and sfix sizes have to be parametrized correctly,
     # such that sfix.k > sfix.f >= sfloat vlen. This is not the case by default.
\end{verbatim}


\msubsubsection{$\mathsf{FLLog2}((\sshare{v_1},\sshare{p_1},\sshare{z_1},\sshare{s_1},\sshare{\err_1}))$:}
  In the following algorithm $(cv_i,cp_i,0,0)$ represents the floating point
  constant $(2 \cdot \log_2 e)/(2 \cdot i+1)$.
  \begin{enumerate}
    \item $M \asn \ceil{ \ell/(2 \cdot \log_2 3)-1/2}$.
    \item $(\sshare{v_2},\sshare{p_2},0,0,0) \asn \mathsf{FLSub}((2^{\ell-1},-(\ell-1),0,0,0),(\sshare{v_1},-\ell,0,0,0))$.
    \item $(\sshare{v_3},\sshare{p_3},0,0,0) \asn \mathsf{FLAdd}((2^{\ell-1},-(\ell-1),0,0,0),(\sshare{v_1},-\ell,0,0,0))$.
    \item $(\sshare{v_y},\sshare{p_y},0,0,0) \asn \mathsf{FLDiv}((\sshare{v_2},\sshare{p_2},0,0,0),(\sshare{v_3},\sshare{p_3},0,0,0))$.
    \item $(\sshare{v_{y^2}},\sshare{p_{y^2}},0,0,0) \asn \mathsf{FLMult}((\sshare{v_y},\sshare{p_y},0,0,0),(\sshare{v_y},\sshare{p_y},0,0,0))$.
    \item $(\sshare{v},\sshare{p},0,0,0) \asn  \mathsf{FLMult}((\sshare{v_y},\sshare{p_y},0,0),(cv_0,cp_0,0,0))$.
    \item For $i \in[1,\ldots,M]$ do
          \begin{enumerate}
            \item $(\sshare{v_y},\sshare{p_y},0,0,0) \asn \mathsf{FLMult}((\sshare{v_y},\sshare{p_y},0,0,0),(\sshare{v_{y^2}},\sshare{p_{y^2}},0,0,0))$.
            \item $(\sshare{v_2},\sshare{p_2},0,0,0) \asn \mathsf{FLMult}((\sshare{v_y},\sshare{p_y},0,0,0),(cv_i,cp_i,0,0,0))$.
            \item $(\sshare{v},\sshare{p},0,0,0) \asn \mathsf{FLAdd}((\sshare{v},\sshare{p},0,0,0),(\sshare{v_2},\sshare{p_2},0,0,0))$.
          \end{enumerate}
  \item $(\sshare{v_2},\sshare{p_2},\sshare{z_2},\sshare{s_2},\sshare{\err_2}) \asn \mathsf{Int2FL}(\ell+\sshare{p},\ell,\ell)$. % Note paper here says $\ell - \sshare{p}$ but this is definitely wrong
    \item $(\sshare{v},\sshare{p},\sshare{z},\sshare{s},\sshare{\err}) \asn \mathsf{FLSub}((\sshare{v_2},\sshare{p_2},\sshare{z_2},\sshare{s_2},\sshare{\err_2}),(\sshare{v},\sshare{p},0,0,0))$.
    \item $\sshare{a} \asn \mathsf{EQ}(\sshare{p_1},-(\ell-1),k)$.
    \item $\sshare{b} \asn \mathsf{EQ}(\sshare{v_1},2^{\ell-1},\ell)$.
    \item $\sshare{z} \asn \sshare{a} \cdot \sshare{b}$.
    \item $\sshare{v} \asn \sshare{v} \cdot(1-\sshare{z})$.
    \item $\sshare{\err} \asn \sshare{\err_1} + \sshare{z_1} + \sshare{s_1}$.
    \item $\sshare{p} \asn \sshare{p} \cdot(1-\sshare{z})$.
    \item Return $(\sshare{v},\sshare{p},\sshare{z},\sshare{s_1},\sshare{\err})$.
  \end{enumerate}


  \msubsection{Trigonometic Functions}
  \todo{These still need implementing for sfloat in the Rust pipeline.}
  All three basic trigonometric functions support inputs of either
  fixed point or floating point precision.
  With the output type being equal to the input type.
  The computation of $\sin(x)$ and $\cos(x)$ are performed
  using polnoymial approximations, with the computation of $\tan(x)$
  done via $\sin(x)/\cos(x)$.
  The basic idea for $\sin(x)$ and $\cos(x)$ is to first reduce
  the argument $x$ into the range $[0,\ldots,2 \pi)$, so
  as to obtain a new argument (which we call $y$)
  We then compute a bit $b_1$ to test as to whether
$y \in [0,\pi)$ or $[\pi,2 \pi)$ (with $0$ being
  the former).
  We then reduce $y$ to $z$ by reducing it into the range
$[0,\pi)$, and compute a bit $b_2$ which says whether
$z$ is in the range $[0,\pi/2)$ or $[\pi/2,\pi)$.
  We finally reduce $z$ into the range $[0,\pi/2)$ resulting
  in $w$.
  Then a polynomial is used to compute
$\sin(w)$ or $\cos(w)$, which means
  we need to now {\em scale} $w$ into the range $[0,1)$
  to obtain $v$.
  %We then derive the final result using the following
  %identities:
  %\begin{align*}
  %	\sin(x) &= \sin(y) = (1-2 \cdot b_1) \cdot \sin(z), %\\
  %	\cos(x) &= \cos(y) = (1-2 \cdot b_1) \cdot \cos(z), %\\
  %	\sin(z) &= (1-b_2) \cdot \sin(w) + b_2\cdot \cos(w), %\\
  %	\cos(z) &= (1-b_2) \cdot \cos(w) - b_2\cdot \sin(w).
  %\end{align*}
  For the polynomial approximations to the basic functions in
  the range $[0,\pi/2)$, where the argument is given as $w = v \cdot \pi/2$
  we use the following approximations from Hart's book \cite{Hart:1978:CA:540084}
  \begin{align*}
    \sin(w) & = v \cdot P_{3307}(v^2), \\
    \cos(w) & = P_{3508}(v^2).
  \end{align*}
  Where we have
  \begin{center}
    \begin{tabular}{|c||c|l||c|l|}
      \hline
         & \multicolumn{2}{c||}{$P_{3307}$} & \multicolumn{2}{c|}{$P_{3508}$}                                                 \\
      \hline
      0  & 1                                & +.15707 96326 79489 66192 31314 989 & 0   & +.99999 99999 99999 99999 99914 771 \\
      1  & 0                                & -.64596 40975 06246 25365 51665 255 & 0   & -.49999 99999 99999 99999 91637 437 \\
      2  & -1                               & +.79692 62624 61670 45105 15876 375 & -1  & +.41666 66666 66666 66653 10411 988 \\
      3  & -2                               & -.46817 54135 31868 79164 48035 89  & -2  & -.13888 88888 88888 88031 01864 15  \\
      4  & -3                               & +.16044 11847 87358 59304 30385 5   & -4  & +.24801 58730 15870 23300 45157     \\
      5  & -5                               & -.35988 43235 20707 78156 5727      & -6  & -.27557 31922 39332 25642 1489      \\
      6  & -7                               & +.56921 72920 65732 73962 4         & -8  & +.20876 75698 16541 25915 59        \\
      7  & -9                               & -.66880 34884 92042 33722           & -10 & -.11470 74512 67755 43239 4         \\
      8  & -11                              & +.60669 10560 85201 792             & -13 & +.47794 54394 06649 917             \\
      9  & -13                              & -.43752 95071 18174 8               & -15 & -.15612 26342 88277 81              \\
      10 & -15                              & +.25002 85418 9303                  & -18 & +.39912 65450 7924                  \\
      \hline
    \end{tabular}
  \end{center}
  NOTE: Polynomial tables are described by the monomial number, the degree of the approximation $p$ and its significand $s$. The coefficient of the $i$th monomial can be obtained by multiplying the significand by $10^{p_{i}}$ as follows: $s_{i} \cdot 10^{p_{i}}$.

  \msubsubsection{$\mathsf{F\star TrigSub}(\sshare{x})$:}
  \begin{enumerate}

    \item $\sshare{f} \asn \mathsf{F\star Mult}(\sshare{x},(1 /(2 \cdot \pi))$
    \item $\sshare{f} \asn \mathsf{F\star Floor}(\sshare{f})$.
    \item $\sshare{y} \asn \mathsf{F\star Mult}(\sshare{f},(2 \cdot \pi ))$.
    \item $\sshare{y} \asn \mathsf{F\star Add}(\sshare{x},-\sshare{y})$.
    \item $\sshare{b_1} \asn \mathsf{F\star GE}(\sshare{y},(\pi))$.
    \item $\sshare{f} \asn \mathsf{F\star Add}(2 \cdot \pi,-\sshare{y})$
    \item $w \asn \mathsf{F\star Choose}(\sshare{f},\sshare{y}, \sshare{b_1})$.
    \item $\sshare{b_2} \asn \mathsf{F\star GE}(\sshare{2},(\pi / 2))$.
    \item $\sshare{f} \asn \mathsf{F\star Add}( \pi,-\sshare{w})$
    \item $w \asn \mathsf{F\star Choose}(\sshare{f},\sshare{w}, \sshare{b_2})$.
    \item Return $(\sshare{w},\sshare{b_1},\sshare{b_2})$.

  \end{enumerate}

  \paragraph{MAMBA Example:} To reduce the angle you could execute the following (note that this function call is meant to be used internally):
  \begin{lstlisting}[language={python}]
from Compiler import mpc_math
x = sfix(4) # sfloat(4)
# returns an angle in the [0,pi/2) interval in w and flags b1 and b2.
w, b1, b2 = mpc_math.sTrigSub_fx(x)
\end{lstlisting}

  \msubsubsection{$\mathsf{F\star Sin}(\sshare{x})$}
  We present these routines as generic routines given the specific helper subroutine above;
  we assume an obvious overloading/translation of arguments.
  We let $\mathsf{F\star Choose}(\sshare{x},\sshare{y},\sshare{b})$, for a
  shared bit $b$, denote an operation which produces $\sshare{x}$ if $\sshare{b}=1$
  and $\sshare{y}$ otherwise
  This is easily obtained by securely multiplying each component share
  of the data representing $\sshare{x}$ etc by $\sshare{b}$.
  So for fixed point representations this becomes, irrespective of the
  values $k$ and $f$,
  \begin{enumerate}
    \item $\sshare{a} \asn \sshare{b}\cdot \sshare{x}+(1-\sshare{b}) \cdot \sshare{y}$
  \end{enumerate}
  For floating point representations this becomes
  \begin{enumerate}
    \item $\sshare{w}, \sshare{b_1}, \sshare{b_2} \asn \mathsf{F\star TrigSub}(\sshare{x})$
    \item $\sshare{v} \asn \sshare{w} \cdot (1/(\pi/2))$.
    \item $\sshare{b} \asn \mathsf{F\star Choose}(\sshare{-1},\sshare{1}, \sshare{b_1})$.
    \item $\sshare{\sin(v)} \asn \sshare{v} \cdot \mathsf{F\star Pol}(P_{3307},\sshare{v^2})$.
    \item Return $(\sshare{b} \cdot \sshare{\sin(v)})$.
  \end{enumerate}
  %
  %Given this we can now compute $\sin(x)$ in the following lines:
  %\begin{enumerate}
  %\item $(\sshare{\sin(w)},\sshare{\cos(w)},\sshare{b_1},\sshare{b_2}) \asn \mathsf{F\star TrigSub}(\sshare{x})$.
  %\item $\sshare{\sin(z)} \asn \mathsf{F\star Choose}(\sshare{\sin(w)},\sshare{\cos(w)},1-\sshare{b_2})$.
  %\item $\sshare{-\sin(z)} \asn \mathsf{F\star Neg}(\sshare{\sin(z)}$.
  %\item $\sshare{\sin(x)} \asn \mathsf{F\star Choose}(\sshare{\sin(z)},\sshare{-\sin(z)},1-\sshare{b_1})$.
  %\item Return $\sshare{\sin(x)}$.
  %\end{enumerate}

  \paragraph{MAMBA Example:} To obtain the \verb|sin| of any value, you could execute the following:
  \begin{lstlisting}[language={python}]
from Compiler import mpc_math
x = sfix(4) # sfloat(4)
# returns the sin of a number of any interval
y = mpc_math.sin(x)
\end{lstlisting}

  \msubsubsection{$\mathsf{F\star Cos}(\sshare{x})$}
  Likewise this becomes
  \begin{enumerate}

    \item $\sshare{w}, \sshare{b_1}, \sshare{b_2} \asn \mathsf{F\star TrigSub}(\sshare{x})$
    \item $\sshare{v} \asn \sshare{w}$.
    \item $\sshare{b} \asn \mathsf{F\star Choose}(\sshare{-1},\sshare{1}, \sshare{b_2})$.
    \item $\sshare{\cos(v)} \asn  \mathsf{F\star Pol}(P_{3308},\sshare{v^2})$.
    \item Return $(\sshare{b} \cdot \sshare{\cos(v)})$.

  \end{enumerate}

  \paragraph{MAMBA Example:} To obtain the \verb|sin| of any value, you could execute the following:
  \begin{lstlisting}[language={python}]
from Compiler import mpc_math
x = sfix(4) # sfloat(4)
# returns the cos of an angle on any interval
y = mpc_math.cos(x)
\end{lstlisting}

  \msubsubsection{$\mathsf{F\star Tan}(\sshare{x})$}
  Likewise this becomes
  \begin{enumerate}

    \item $(\sshare{w},\sshare{b_1},\sshare{b_2}) \asn \mathsf{F\star TrigSub}(\sshare{x})$.
    \item $\sshare{v} \asn \sshare{w} \cdot (1/(\pi/2))$.
    \item $\sshare{b} \asn \mathsf{F\star Choose}(\sshare{-1},\sshare{1}, \sshare{b_1})$.
    \item $\sshare{\sin(v)} \asn \sshare{v} \cdot \mathsf{F\star Pol}(P_{3307},\sshare{v^2})$.
    \item $\sshare{\sin(x)} \asn (\sshare{b} \cdot \sshare{\sin(v)})$.
    \item $\sshare{v} \asn \sshare{w}$.
    \item $\sshare{b} \asn \mathsf{F\star Choose}(\sshare{-1},\sshare{1}, \sshare{b_2})$.
    \item $\sshare{\cos(v)} \asn  \mathsf{F\star Pol}(P_{3308},\sshare{v^2})$.
    \item $\sshare{\sin(x)} \asn (\sshare{b} \cdot \sshare{\cos(v)})$.
    \item $\sshare{\tan(x)} \asn \mathsf{F\star Div}(\sshare{\sin(x)},\sshare{\cos(x)})$.
    \item Return $\sshare{\tan(x)}$.
  \end{enumerate}

  \paragraph{MAMBA Example:} To obtain the \verb|tan| of any value, you could execute the following:
  \begin{lstlisting}[language={python}]
from Compiler import mpc_math
x = sfix(4) # sfloat(4)
# returns the tan of an angle on any interval
y = mpc_math.tan(x)
\end{lstlisting}

  \msubsection{Inverse Trigonometric Functions}
  \todo{Given that MAMBA currently only supports square root operations for sfix inputs,
    inverse trigonometric functions are restricted to sfix inputs.}
  \todo{These still need implementing for sfloat in the Rust pipeline.}
  To obtain  $\arcsin$ and $\arccos$ one makes use of the formula:
  \begin{align*}
    \arcsin (x) & = \arctan \left( \frac{x}{\sqrt{1-x^2}} \right), \\
    \arccos (x) & = \frac{\pi}{2} - \arcsin (x).
  \end{align*}
  Note, that $\arcsin$ and $\arccos$ are only defined when
$|x|\le 1$.
  The value of $\arctan(x)$ is however defined for all real $x$.
  For $\arctan(x)$ we first reduce to positive values of $x$ by using the formula
  \[ \arctan(-x) = - \arctan(x). \]
  We then reduce to the interval $[0,1)$ using the formula
  \[ \arctan(x) = \frac{\pi}{2} - \arctan\left(\frac{1}{x} \right). \]
  The final approximation to $\arctan(x)$ for $x \in [0,1)$
  is obtained using the Pade approximation $P_{5102}/Q_{5102}$
from Hart's book \cite{Hart:1978:CA:540084}.
Where the polynomials are represented as in our earlier
descriptions.
\begin{center}
  \begin{tabular}{|c||c|l||c|l|}
    \hline
      & \multicolumn{2}{c||}{$P_{5102}$} & \multicolumn{2}{c|}{$Q_{5102}$}                                               \\
    \hline
    0 & 5                                & +.21514 05962 60244 19331 93254 468 & 5 & +.21514 05962 60244 19331 93298 234 \\
    1 & 5                                & +.73597 43380 28844 42408 14980 706 & 5 & +.80768 78701 15592 48851 76713 209 \\
    2 & 6                                & +.10027 25618 30630 27849 70511 863 & 6 & +.12289 26789 09278 47762 98743 322 \\
    3 & 5                                & +.69439 29750 03225 23370 59765 503 & 5 & +.97323 20349 05355 56802 60434 387 \\
    4 & 5                                & +.25858 09739 71909 90257 16567 793 & 5 & +.42868 57652 04640 80931 84006 664 \\
    5 & 4                                & +.50386 39185 50126 65579 37791 19  & 5 & +.10401 13491 56689 00570 05103 878 \\
    6 & 3                                & +.46015 88804 63535 14711 61727 227 & 4 & +.12897 50569 11611 09714 11459 55  \\
    7 & 2                                & +.15087 67735 87003 09877 17455 528 & 2 & +.68519 37831 01896 80131 14024 294 \\
    8 & -1                               & +.75230 52818 75762 84445 10729 539 & 1 & +.1                                 \\
    \hline
  \end{tabular}
\end{center}

The following protocol for \verb|arcsin| implements the formulas from before.
The protocol and its implementation, make use of our secure implementation of Square Root.
This method is implemented, and it is used to derive \verb|arccos|.

\msubsubsection{$\mathsf{F\star ArcSin}(\sshare{x})$}
\begin{enumerate}
  \item $\sshare{x^2} \asn \mathsf{F\star Mult}(\sshare{x},\sshare{x})$.
  \item $\sshare{-x^2} \asn \mathsf{F\star Neg}(\sshare{x^2})$.
  \item $\sshare{1-x^2} \asn \mathsf{F\star Add}(1,\sshare{-x^2})$.
  \item $\sshare{\sqrt{1-x^2}} \asn \mathsf{F\star Sqrt}(1,\sshare{1-x^2})$.
  \item $\sshare{v} \asn \mathsf{F\star Div}(\sshare{x},\sshare{\sqrt{1-x^2}})$.
  \item $\sshare{y} \asn \mathsf{F\star ArcTan}(\sshare{v})$.
  \item If $\star=\mathsf{L}$
        \begin{enumerate}
          \item $\sshare{|x|} \asn \mathsf{FLAbs}(\sshare{x})$.
          \item $\sshare{y_\err} \asn \mathsf{FLGT}(\sshare{|x|},1.0)$.
        \end{enumerate}
  \item Return $\sshare{y}$.
\end{enumerate}


\paragraph{MAMBA Example:} To obtain the \verb|arcSin| on the  (-1,1) interval, you could execute the following:
\begin{lstlisting}[language={python}]
from Compiler import mpc_math
x =sfix(0.5)
# returns the tan of an angle on any interval
y = mpc_math.asin(x)
\end{lstlisting}


\msubsubsection{$\mathsf{F\star ArcCos}(\sshare{x})$}
\begin{enumerate}
  \item $\sshare{y} \asn \mathsf{F\star ArcSin}(\sshare{x})$.
  \item $\sshare{-y} \asn \mathsf{F\star Neg}(\sshare{y})$.
  \item $\sshare{\pi/2-y} \asn \mathsf{F\star Add}(\pi/2,\sshare{-y})$.
  \item Return $\sshare{y}$.
\end{enumerate}

\paragraph{MAMBA Example:} To obtain the \verb|arcCos| of any value on the (-1,1) interval, you could execute the following:
\begin{lstlisting}[language={python}]
from Compiler import mpc_math
x =sfix(0.5)
# returns the tan of an angle on any interval
y = mpc_math.acos(x)
\end{lstlisting}

\msubsubsection{$\mathsf{F\star ArcTan}(\sshare{x})$}
\begin{enumerate}
  \item $\sshare{s} \asn \mathsf{F\star LTZ}(\sshare{z})$.
  \item $\sshare{|x|} \asn \mathsf{F\star Abs}(\sshare{x})$.
  \item $\sshare{b} \asn \mathsf{F\star GT}(\sshare{|x|},1.0)$.
  \item $\sshare{v} \asn \mathsf{F\star Div}(1,\sshare{|x|})$.
  \item $\sshare{v} \asn \mathsf{F\star Choose}(\sshare{|x|},\sshare{v},1-\sshare{b})$.
  \item $\sshare{v^2} \asn \mathsf{F\star Mul}(\sshare{v},\sshare{v})$.
  \item $\sshare{y} \asn \mathsf{FxPade}(P_{5102},Q_{5102},\sshare{v^2})$.
  \item $\sshare{y} \asn \mathsf{F\star Mul}(\sshare{v},\sshare{y})$.
  \item $\sshare{\pi/2-y} \asn \mathsf{F\star Sub}(\pi/2,\sshare{y})$.
  \item $\sshare{y} \asn \mathsf{F\star Choose}(\sshare{y},\sshare{\pi/2-y},1-\sshare{b})$.
  \item $\sshare{-y} \asn \mathsf{F\star Neg}(\sshare{y})$.
  \item $\sshare{y} \asn \mathsf{F\star Choose}(\sshare{y},\sshare{-y},1-\sshare{s})$.
  \item Return $\sshare{y}$.
\end{enumerate}

\paragraph{MAMBA Example:} To obtain the \verb|arcTan| of any value, you could execute the following:
\begin{lstlisting}[language={python}]
from Compiler import mpc_math
x =sfix(0.5)
# returns the tan of an angle on any interval
y = mpc_math.atan(x)
\end{lstlisting}
