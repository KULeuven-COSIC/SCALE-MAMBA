
\mainsection{Changes}
This section documents changes made since the ``last'' release of
the software. 

\vspace{5mm}

\noindent
This is currently {\bf version 1.3} of the SCALE and MAMBA software.
There are two main components to the system, a run time system called
SCALE,
\begin{center}
  Secure Computation Algorithms from LEuven
\end{center}
and a Python-like programming language called MAMBA
\begin{center}
  Multiparty AlgorithMs Basic Argot
\end{center}
Note, that MAMBA is a type of snake (like a Python), a snake
has scales, and an argot is a ``secret language''.

\vspace{5mm}

\noindent
The software is a major modification of the earlier SPDZ software. 
You {\em should not} assume it works the same so please read
the documentation fully before proceeding.

\subsection{Changes in version 1.3 from 1.2}
\begin{enumerate}
\item New offline phase for full threshold called TopGear has been
implemented \cite{TopGear}. This means we have changed a number
of FHE related security parameters. In particular the new parameter
sets are {\em more} secure.
{\bf But} they are likely to be different from the ones you have
been using. Thus you will need to generate again FHE parameters
(using \verb+Setup.x+).
\item Bug fixed when using Shamir with a large number of parties
and very very low threshold value.
\item Renaming of some bytecodes to make their meaning clearer.
\item Floating point operations are now (almost) fully supported.
We have basic operations on floating point \verb+sfloat+ variables
supported, as well as basic trigonometric functions. Still to
be completed are the square root, logarithm and exponential functions.
Note that, the implementation of floating point numbers is
different from that in the original SPDZ compiler. The main alteration
is an additional variable to signal error conditions within the
number, and secure processing of this signal.
\end{enumerate}


\subsection{Changes in version 1.2 from 1.1}

\begin{enumerate}
\item A lot of internal re-writing to make things easier to maintain.
\item There are more configuration options in \verb+config.h+ to enable;
      e.g. you can now give different values to the different places
      we use statistical security parameters.
\item Minor correction to how FHE parameters are created. This means
      that the FHE parameters are a bit bigger than they were before.
      Probably a good idea to re-run setup to generate new keys etc
      in the Full Threshold case.
\item Minor change to how \verb+CRASH+ works. If the IO handler returns, 
      then the \verb+RESTART+ instruction is automatically called.
\item There is a new run time switch \verb+maxI+ for use when performing
      timings etc. This is only for use when combined with the \verb+max+
      switch below.
\item Two new instructions \verb+CLEAR_MEMORY+ and \verb+CLEAR_REGISTERS+
      have been added. 
      These are called from MAMBA via \verb+clear_memory()+ and \verb+clear_registers()+.
      The first can only be called with compilation is done with the flag \verb+-M+, whilst
      the second {\em may} issue out of order (consider it experimental at present).
\item Bug fix for \verb+sfix+ version of \verb+arcsin/arccos+ functions.
\item When running Shamir with a large number of parties we now move to using
      an interactive method to produce the PRSS as opposed to the non-interactive
      method which could have exponential complexity. This means we can now cope
      with larger numbers of parties in the Shamir sharing case. An example
      ten party example is included to test these routines.
\end{enumerate}


\subsection{Changes in version 1.1 from 1.0}

\begin{enumerate}
\item Major bug fix to IO processing of private input and output.
This has resulted in a change to the bytecodes for these instructions.
\item We now support multiple FHE Factory threads, the precise number
is controlled from a run-time switch.
\item The restart methodology now allows the use to programmatically
create new schedules and program tapes if so desired. The
``demo'' functionality is however exactly as it was before.
Please see the example functionality in the file
\verb+src/Input_Output/Input_Output_Simple.cpp+
\item We have added in some extra verbose output to enable timing of the
offline phase. To time the offline phase, on say bit production,
you can now use the program \verb+Program/do_nothing.mpc+, and then
execute the command for player zero.
\begin{verbatim}
   ./Player.x -verbose 1 -max 1,1,10000000  0 Programs/do_nothing/
\end{verbatim}
Note square production on its own is deliberately throttled
so that when run in a real execution bit production is preferred
over squares.
By altering the constant in the program  \verb+Program/do_nothing.mpc+
you can also alter the number of threads used for this timing operation.
If you enter a negative number for verbose then verbose output
is given for the online phase; i.e. it prints the bytecodes being
executed.
\item The fixed point square root functions have now been extended to
cope with full precision fixed point numbers.
\item The \verb+PRINTxxx+ bytecodes now pass their output via the
\verb+Input_Output+ functionality. These bytecodes are meant for
debugging purposes, and hence catching them via the IO functionality
makes most sense. The relevant function in the IO class
is \verb+debug_output+.
\item We have added the ability to now also input and output \verb+regint+
values via the IO functionality, with associated additional bytecodes
added to enable this.
\item The IO class now allows one to control the opening and closing of
channels, this is aided by two new bytecodes for this purpose
called \verb+OPEN_CHAN+ and \verb+CLOSE_CHAN+.
\item Input of clear values via the IO functionality (i.e.
for \verb+cint+ and \verb+regint+ values) is now internally checked to ensure that
all players enter the same clear values.
Note, this requires modification to any user defined 
derived classes from \verb+Input_Output_Base+.
See the chapter on IO for more details on this.
\item The way the chosen IO functionality is bound with the main program
has now also been altered. See the chapter on IO for more details on this.
\item These changes have meant there are a number of changes to the
specific byte codes, so you will need to recompile MAMBA programs.
If you generate your own bytecodes then your backend will need to
change as well.
\end{enumerate}


\subsection{Changes From SPDZ}
Apart from the way the system is configured and run there are
a number of functionality changes which we outline below.

\subsubsection{Things no longer supported}
\begin{enumerate}
\item We do not support any $GF(2^n)$ arithmetic in the run time 
environment. The compiler will no longer compile your programs.
\item There are much fewer switches in the main program, as we want
to produce a system which is easier to support and more useful in
building applications.
\item Socket connections, file, and other forms of IO to the main MPC
engine is now unified into a single location. This allows {\em you} to
extend the functionality without altering the compiler or
run-time in any way (bar changing which IO class you load
at compile time). See Section \ref{sec:IO} for details.
\end{enumerate}

\subsubsection{Additions}
\begin{enumerate}
\item The offline and online phases are now fully integrated.
This means that run-times will be slower than you would have got
with SPDZ, but the run-times obtained are closer to what you would
expect in a ``real'' system.
{\bf Both} the online and offline phases are {\bf actively} secure
with abort.
\item Due to this change it can be slow to start a new instance
and run a new program. So we provide a new (experimental) operation which 
``restarts'' the run-time. This is described in Section \ref{sec:restart}.
This operation is likely to be revamped and improved in the next
release as we get more feedback on its usage.
\item We support various Q2 access structures now, which can be
defined in various ways: Shamir threshold, via Replicated sharing,
or via a general Monotone Span Programme (MSP).
For replicated sharing you can define the structure
via either the minimally qualified sets, or the maximally unqualified
sets.
For general Q2-MSPs you can input a non-multiplicative MSP and the
system will find an equivalent multiplicative one for you using the
method of \cite{CDM00}.
\item Offline generation for Q2 is done via Maurer's method \cite{Maurer}, but
for Replicated you can choose between Maurer and the reduced communication
method of Keller, Rotaru, Smart and Wood \cite{KRSW}.
For general Q2-MSPs, and Shamir sharing, the online phase is the method described in Smart and Wood \cite{SW18},
with ({\em currently}) the offline phase utilizing Maurer's multiplication method \cite{Maurer}.
\item All player connections are now via SSL, this is not strictly
needed for full threshold but is needed for the other access structures
we now support.
\item We now have implemented more higher level mathematical functions for
the \verb+sfix+ datatype, and corrected a number of bugs. 
A similar upgrade is expected in the next release for the \verb+sfloat+ type.
\end{enumerate}

\subsubsection{Changes}
\begin{enumerate}
\item The {\bf major} change is that the offline and online phases are now integrated.
This means that to run quick test runs, using full threshold is going to take
ages to set up the offline data. Thus for test runs of programs in the online
phase it is best to test using one of the many forms of Q2 access structures.
For example by using Shamir with three players and threshold one. Then once your
online program is tested you can move to a production system with two players
and full threshold if desired.
\item You now compile a program by executing
\begin{verbatim}
      ./compile.py Programs/tutorial
\end{verbatim}
where \verb+Programs/tutorial+ is a {\em directory} which contains
a file called \verb+tutorial.mpc+. Then the compiler puts all of the
compiled tapes etc {\em into this directory}.
This produces a much cleaner directory output etc.
By typing \verb+make pclean+ you can clean up all pre-compiled directorys
into their initial state.
\item The compiler picks up the prime being used for secret sharing
after running the second part of \verb+Setup.x+. So you need to recompile
the \verb+.mpc+ files if  you change the prime used in secret sharing, and you
should not compile any SCALE \verb+.mpc+ programs before running \verb+Setup.x+.
\item Internally (i.e. in the C++ code), a lot has been re-organized. The major simplification
is removal of the \verb+octetstream+ class, and it's replacement by a combination
of \verb+stringstream+ and \verb+string+ instead. This makes readibility
much easier.
\item All opcodes in the range \verb+0xB*+ have been changed, so any byte codes
you have generated from outside the python system will need to be changed.
\item We have tried to reduce dependencies between classes in the 
C++ code a lot more. Thus making the code easier to manage.
\item Security-wise we use the latest FHE security estimates for the
FHE-based part, and this can be easily updated. See Chapter \ref{sec:fhe} on FHE
security later.
\end{enumerate}


