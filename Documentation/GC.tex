\mainsection{User Defined Circuits for Garbling}
\label{sec:GC}
SCALE-MAMBA allows one to call user defined circuits, both
from the byte-code and the MAMBA language.
In this section we explain how this is done.

\subsection{Defining Circuits}
The first task is to define the actual circuit.
For this we use a flat file format which we call 
``Bristol Fashion''.
For the precise details of this format see the
file 
\begin{center}
\verb|src/GC/Circuit.h|
\end{center}
To create such files you can use any tool you want,
but here is how we do ours....

If you look in the directory \verb|Circuits/VHDL| you
will find VHDL code for various functions. Write your
VHDL code and compile it to a netlist using whatever
tool chain you have.
The key point is that {\em inputs} and {\em outputs}
to the function must be a multiple of 64-bits in length.
If this is not the case you just need to pad the
input/output to the correct size.

You then need to convert the netlist into ``Bristol 
Fashion''. To do this we use a small program
called \verb|convert.x| which lives in the 
\verb|Circuits| directory. This takes a \verb|.net|
file in the \verb|VHDL| subdirectory and turns it
into a \verb|.txt| file in the \verb|Bristol| subdirectory.

{\bf Note:} Some optimizations are applied to the
circuit on this transfer, which include removing unused
wires etc. If your original function needed padding to
make 64-bit multiple inputs/outputs you need to stop
this optimization. To do this just comment out the
line \verb|SC.Simplify()| in the program \verb|convert.cpp|.

You should now have a Bristol Fashion file with an 
extension \verb|.txt| in the directory \verb|Circuits/Bristol|.
To test this file does what you expect it to do you
can use similar code to that which is given in the
program \verb|Test/Test-Circuit.cpp| or
\verb|Test/Test-Convert.cpp|.

\subsection{Adding Circuits into the RunTime Engine}
So given your new circuit, which we will pretend is
called \verb|Foo.txt|, you now need to register this
with the run time system.
All circuits are given a number by the system,
the user defined numbers are from 65536 upwards (inclusive),
with numbers for the developers being those less
than 65536.
So {\em do not use a number less than 65536!!!!!}.

Circuits which are used to operate on the basic \verb|sregint|
data type to enable 64-bit secret arithmetic are
given numbers less than 100.
Numbers in the range 100 to 65535 are used to add extra
cool (well we think it is cool) functionality into the
system.
We have added a number of such circuits already into the
system, which include AES operations (for AES-128, AES-192
and AES-256) plus a circuit for the Keccak-f function\footnote{If
you have a cool circuit which you think others might find
cool to use in SCALE-MAMBA, just send it to us and we
might include it in a future release with a circuit
number less than 65536.}.

To register your circuit with the runtime you edit the
file \verb|src/GC/Base_Circuits.cpp|. At the bottom of
the initialize member function you can add your
own circuits.
Thus to add your \verb|Foo.txt| circuit you would add
the lines
\begin{lstlisting}
  loaded.insert(make_pair(65536, false));
  location.insert(make_pair(65536, "Circuits/Bristol/Foo.txt"));
\end{lstlisting}
What this does is tell the runtime that we are going to
use a new circuit numbered 65536, that it is not yet loaded
into memory, and that when it does need to be loaded
where to get it from.

\subsection{Byte Code Operation}
The garbled circuit is executed by the \verb|GC| byte-code.
This takes as input three integers; the first specifies
the circuit number to execute, the second defines
the number of \verb|sregint| registers which
give the output, whilst the third defines the number of
\verb|sregint| registers which give the input.
Since a \verb|sregint| register is 64-bits long, this
explains why the circuits take inputs/outputs a multiple
of 64-bits.
After these three integers the instruction takes the list
of output registers, followed by the input registers.

When this instruction is executed, if the circuit has
not yet been loaded it is loaded into memory. Then the
Garbled Circuit engine is called with the given input
registers to produce the output, which is places in the
given output registers.
Note, that the first time a GC operation is met, the
runtime might give a small delay as the various pre-processing
queues need to fill up.
In subsequent GC operations this stall disappears (unless
you are doing a lot of GC operations one after another).

Due to the way the queues are produced if you pass in a
VERY big circuit the  runtime might abort. If this happens
please let us know and we will fix this. But as we never
meet this limit we have a cludge in place to just catch
the issue and throw an exception. The exception says
to contact us, so you will know which one it is.

\subsection{Using Circuits From MAMBA}
MAMBA calling of the circuits follows much like the
\verb|GC| byte-code. The key is that you need to
pass in an array of \verb|sregint| registers, and that
the output registers need to have been created before
the function is called.
In the context of executing AES-128 operation with
the all zero key and the all one message, this becomes
the code
\begin{lstlisting}
AES_128=100

# Set key
key = [sregint(0) for _ in range(2)]

# Set message
mess = [sregint(-1) for _ in range(2)]

# Create space for the ciphertext
ciph = [sregint() for _ in range(2)]

print_ln("AES-128")
# Op, # Output, # Input
GC(AES_128, 2, 4, *(ciph + key + mess))

# Now open the values to check all is OK
m = [None] * 2
k = [None] * 2
c = [None] * 2
for i in range(2):
	m[i] = mess[i].reveal()
	k[i] = key[i].reveal()
	c[i] = ciph[i].reveal()

print_ln("Key")
k[1].public_output()     
k[0].public_output()

print_ln("Message")
m[1].public_output()     
m[0].public_output()

print_ln("Ciphertext")
c[1].public_output()     
c[0].public_output()
\end{lstlisting}
Note, that the AES-128 operation takes four
input registers (two for the key and two for
the message), and produces two output registers
as output.
Also note bit ordering of the inputs and outputs.
The correct output of AES-128 for all zero key
and all one message is
\begin{center}
  \verb|3F5B8CC9EA855A0AFA7347D23E8D664E|
\end{center}
with the top 64-bits \verb|3F5B8CC9EA855A0A|
being returns in \verb|c[1]| and the bottom
64-bits \verb|FA7347D23E8D664E| being returned
in \verb|c[0]|.


\subsection{Current System Defined Circuits}
The current list of system defined circuits are
\begin{center}
\begin{tabular}{c|l}
Number & Function \\
\hline
100 & AES-128 \\
101 & AES-192 \\
102 & AES-256 \\
103 & Keccak-f \\
\end{tabular}
\end{center}

