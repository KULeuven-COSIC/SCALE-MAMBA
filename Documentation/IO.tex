\mainsection{New IO Class}
\label{sec:IO}

A major change in SCALE over SPDZ is the way input and output
is handled in the system to an outside data source/sync.
In SPDZ various hooks were placed within the compiler/bytecode
to enable external applications to connect. This resulted in
code-bloat as the run-time had to support all possible
ways someone would want to connect.

In SCALE this is simplified to a simple IO procedure for
getting data in and out of the MPC engine. It is then up
to the application developer to write code (see below)
which catches and supplies this data in the way that is
wanted by their application.
This should be done without needing to modify then bytecode,
runtime system, or any compiler being used.

\subsection{Adding your own IO Processing}
We identify a number of different input/output scenarios
which are captured in the C++ abstract class 
\begin{verbatim}
    src/Input_Output/Input_Output_Base.h
\end{verbatim}
To use this interface, an application programmer will have to
write their own derived class, just like in the example
class
\begin{verbatim}
    src/Input_Output/Input_Output_Simple.h
\end{verbatim}
given in the distribution.
Then to compile the system they will need to place the
name of their derived class in the file
\begin{verbatim}
    src/config.h
\end{verbatim}
in the macro variable \verb+IO_FUNCTIONALITY+.
Finally any ``configuration'' for this derived class needs
to be placed in the file
\begin{verbatim}
    src/Player.cpp
\end{verbatim}
in the line which initializes the IO class, which for
our simple demo example is given by
\begin{verbatim}
    machine.IO.init(cin,cout,true)
\end{verbatim}
Internally the IO class can maintain any number of ``channels''
for each of the various operations below.
The runtime bytecodes can then pass the required channel to the 
IO class; if no channel is specified in the MAMBA language
then channel zero is selected by default (although for
the \verb+input_shares+ and \verb+output_shares+ commands
you {\em always} need to specify a channel.

\subsection{Types of IO Processing}
In this section we outline the six forms of input and
output supported by the new IO system.

\subsubsection{Private Output}
To obtain an $\F_p$ element privately to one party one
executes the bytecodes
\begin{center}
\verb+START_PRIVATE_OUTPUT+ and \verb+STOP_PRIVATE_OUTPUT+.
\end{center}
These correspond to the old SPDZ bytecodes
\verb+STARTPRIVATEOUTPUT+ and \verb+STOPPRIVATEOUTPUT+.
They enable a designated party to obtain output of
one of the secretly shared variables, this is then passed
to the sytem by a call to the function
\verb+IO.private_output_gfp(const gfp& output,unsigned int channel)+ in the C++ IO class.

\subsubsection{Private Input}
This is the oppposite operation to the one above and it 
is accomplised by the bytecodes 
\begin{center}
\verb+START_PRIVATE_INPUT+ and \verb+STOP_PRIVATE_INPUT+.
\end{center}
These correspond to the old SPDZ bytecodes
\verb+STARTINPUT+ and \verb+STOPINPUT+.
They enable a designated party to enter a value into
the computation.
The value that the player enters is obtained via a call to the 
member function
\verb+IO.private_input_gfp(unsigned int channel)+ in the C++ IO class.

\subsubsection{Public Output}
To obtain public output, i.e. the output of an opened variable,
then the bytecode is \verb+OUTPUT_CLEAR+, corresponding to
old SPDZ bytecode \verb+RAWOUTPUT+.
This output needs to be caught by the C++ IO class in
the member function \verb+IO.public_output_gfp(const gfp& output,unsigned int channel)+.

\subsubsection{Public Input}
A clear public input value is something which is input to
{\em all} players; and must be the same input for all players.
This hooks into the C++ IO class function 
\verb+IO.public_input_gfp(unsigned int channel)+, and corresponds to the bytecode
\verb+INPUT_CLEAR+.
This is a bit like the old SPDZ bytecode \verb+PUBINPUT+
but with slightly different semantics.

\subsubsection{Share Output}
In some situations the system might want to store some
internal state, in particular the shares themselves.
To enable this we provide the \verb+OUTPUT_SHARE+
bytecode (corresponding roughly to the old 
\verb+READFILESHARE+ bytecode). The IO class can do
whatever it would like with this share obtained. However,
one needs to be careful that any usage does not break the
MPC security model.
The member function hook to deal with this type of
output is the function
\verb+IO.output_share(const Share& S,unsigned int channel)+.

\subsubsection{Share Input}
Finally, shares can be input from an external source
(note they need to be correct/suitably MAC'd). This
is done via the bytecode \verb+INPUT_SHARE+ and the
member function \verb+IO.input_share(unsigned int channel)+.

\subsubsection{Trigger}
There is a special function \verb+IO.trigger()+ which is
used for the parties to signal that they are content with
performing a \verb+RESTART+ command.
See Section \ref{sec:restart} for further details.


\subsection{MAMBA Hooks}
These functions can be called from the MAMBA language via,
for \verb+a+ of type \verb+sint+ and \verb+b+ of type \verb+cint+.
\begin{verbatim}
    # Private Output to player 2 on channel 0 and on channel 11
    a.reveal_to(2)
    a.reveal_to(2,11)

    # Private Input from player 0 on channel 0 and on channel 15
    a=sint.get_private_input_from(0)
    a=sint.get_private_input_from(0,15)

    # Public Output on channel 0 and on channel 5
    b.public_output()
    b.public_output(5)

    # Public Input on default channel 0 and on channel 10
    b=public_input()
    b=public_input(10)

    # Share Output on channel 1000
    output_shares(1000,[sint(1)])

    # Share Input on channel 2000
    inp=[sint()]
    input_shares(2000,*inp)
\end{verbatim}
The \verb+IO.trigger()+ is called only when a \verb+restart()+ command
is executed from MAMBA.

